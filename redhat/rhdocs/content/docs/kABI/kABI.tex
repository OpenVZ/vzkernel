\documentclass[10pt,oneside,english]{book}
\usepackage{mathptmx}
\usepackage{helvet}
\usepackage{courier}
\usepackage[T1]{fontenc}
\usepackage[utf8]{inputenc}
\usepackage[letterpaper]{geometry}
\geometry{verbose,tmargin=1in,bmargin=1in,lmargin=1in,rmargin=1in}
\setcounter{secnumdepth}{1}
\setcounter{tocdepth}{1}
\setlength{\parskip}{\medskipamount}
\setlength{\parindent}{0pt}
\usepackage{babel}
\usepackage{fancybox}
\usepackage{calc}
\usepackage{url}
\usepackage[unicode=true,
 bookmarks=true,bookmarksnumbered=true,bookmarksopen=true,bookmarksopenlevel=2,
 breaklinks=true,pdfborder={0 0 1},backref=section,colorlinks=false]
 {hyperref}
\hypersetup{pdftitle={A Kernel Developer's Guide to kABI},
 pdfauthor={Kernel: Platform Enablement Team},
 pdfkeywords={kernel, kABI}}

\newcommand*{\fullref}[1]{\hyperref[{#1}]{\autoref*{#1} \nameref*{#1}}}

\usepackage{listingsutf8}
\lstdefinestyle{customc}{
  language=C,
  basicstyle=\footnotesize\ttfamily,
  showstringspaces=false,
}
\lstset{style=customc}

\makeatletter

\makeatother

\begin{document}

\title{A Kernel Developer's Guide to kABI}

\author{Kernel: Platform Enablement Team Red Hat, Inc.}
\maketitle

\newlength{\mpwidth}
\setlength{\mpwidth}{1\columnwidth - 2\fboxsep - 2\fboxrule - \shadowsize}

\section*{Abstract}

There can be a burden associated with maintaining RHEL's kABI (Kernel
Application Binary Interface) commitment that persists throughout
the release's entire lifecycle if subsystem maintainers don't pro-actively
accommodate for upstream back-ports. RHEL kernel developers need to
be aware of, and fully comprehend, such restrictions, which is the
focus of this document.

\textbf{Keywords:} kABI, KABI, KMI, Kernel Application Binary Interface,
Kernel Module Interface.

\tableofcontents{}

\chapter{Overview}

There are Red Hat partners and customers that want a means to deliver
and receive updated drivers in a manner that allows continued operation
across subsequent kernel updates. Out-of-band (OOB) driver modules
are the most common, which includes updates such as what the Driver
Update Program (DUP) provides. \footnote{Linux man page: \texttt{kmod(8)}.}
Driver updates enable ``stop gap'' solutions for situations such
as supporting new hardware, or new, or maturing, functionality, prior
to such updates becoming available in a future kernel update.

Extending this concept, a subset of ISV's have said they need the
ability to build, and ship, their own, ``\emph{third party'' }modules,
independently of Red Hat Enterprise Linux (RHEL). These module types
tend not to be open-source, but critical to various partners' proprietary
solutions -- not drivers per say, but more filesystem, or system
management, oriented -- and will never go into a RHEL release.

These customer driven use cases necessitate the need for some level
of binary compatibility with respect to the kernel. To meet these
needs, Red Hat has publicly stated that it will provide ``kernel
ABI stability'', more commonly referred to as the ``\emph{kernel
ABI}'' (\emph{kABI}), along minor updates of a RHEL release but not
across major releases.

Partners cite Red Hat's kABI commitment as a key differentiator, distinguishing
it from other distributions.

Most of the effort concerning kABI derives from third party modules.
The specific type of module -- OOB, or third party -- isn't really
important. What does matter is whether a module is linked with the
kernel statically or dynamically. Modules are strongly tied to the
data structures and function prototypes defined in a particular kernel
version; the interface seen by a module can change significantly from
one kernel version to the next.

Statically linked modules -- ``\emph{built-in}'' or ``\emph{internal
modules}'' -- are incorporated as part of the kernel so most, if
not all, of the concerns related to whether or not the module is compatible
with the specific kernel version -- ``\emph{Kernel/module variance}''
-- don't come into play. Dynamically linked modules -- ``\emph{external
modules}''; sometimes referred to as \emph{out-of-tree} modules --
are stored within the filesystem as separate ELF object files. In
order to run an external module on a kernel the '\texttt{insmod(8)}'
program is executed which compares the globally exported symbols used
by the module against the kernel's symbol's versions and accepts,
or rejects, linking the module to the kernel. Thus, for external modules,
kernel/module variance concerns are paramount.

kABI's role is to proactively disallow changes related to exported
symbol's ABI from occurring.

Red Hat's partners wanting a commitment that they will not have to
rewrite, or recompile, their modules identify the external symbols
used within them and submit a request to Red Hat for protection. Such
requests then have to be reviewed, and explicitly agreed to, by Red
Hat's kernel development team before the symbols are considered protected.

Maintaining kABI protected symbols can cause severe restrictions with
respect to bringing in new content unless additions are made to the
kernel to specifically accommodate future back-porting efforts to
a RHEL release.

\chapter{Technical Details: Linux Kernel - Modules}

\section{Module Versioning}

The enabling of \texttt{CONFIG\_MODVERSIONS} in a Linux kernel's configuration
enforces compatibility of the Kernel Module Interface (KMI), more
commonly referred to as ``\emph{module versioning}''. \footnote{./Documentation/kbuild/modules.txt}
\footnote{Option \texttt{CONFIG\_MODVERSIONS} is enabled in all RHEL configurations.}
This causes all kernel global symbols exported with the \texttt{EXPORT\_SYMBOL()}
and \texttt{EXPORT\_SYMBOL\_GPL()} macros to be augmented with 8-hex-digit
checksums computed from the names of all relevant C tokens that define
the interface. \footnote{./include/linux/export.h: \texttt{EXPORT\_SYMBOL\{\_GPL\}()}}

The checksumming augmentation -- \emph{genksyms} \emph{symbol} \emph{versioning
information} -- is calculated by expanding the symbol to its full
definition where all structures, unions, enums, and typedefs are expanded
down to their basic parts, recursively. This final string is then
used as input to a CRC algorithm yielding an integer that will change
when any constituent part(s) of the symbol's included definitions
change. \footnote{./Makefile, ./scripts/genksyms/{*}: genksyms symbol versioning is
dynamically enabled during a kernel build via the \texttt{\_\_GENKSYMS\_\_}
pre-processing directive. When symbol versioning is needed, \texttt{\_\_GENKSYMS\_\_}
is defined, and the 'genksyms' script is invoked.

./scripts/Makefile.build: \texttt{\$(CPP) -D\_\_GENKSYMS\_\_ \$(c\_flags)
\$< \$(GENKSYMS) ...}}

Note that it's not just the globally exported symbol's names themselves
that are versioned. Encompassed is everything associated with the
symbol's full prototype: ``\emph{all relevant C tokens that essentially
define the interface}''.

Whenever there is a change to an exported symbol's return type, name,
any of its argument's type, or to the type of any field of any structure
contained in, or pointed to, by any of its arguments, including all
data structures within data structures (recursively), then its checksum
changes. When a module attempts loading, each symbol's CRC value contained
in the kernel's symbol table is compared with the corresponding symbol's
CRC value in the module's object file, and if they are not equal the
kernel refuses to load the module. Thus, kernel interface changes
cause the 'insmod' of previously built modules to fail.

Kernel data structures are of particular interest with respect to
KMI as they contain multiple members, some of which may be other data
structures themselves, all of which become recursively included as
part of the symbol's versioning checksum, making the total scope of
what must be preserved in order for an existing module to continue
to load and run, potentially quite large.

As a concrete example, consider the exported symbol '\texttt{pci\_bus\_read\_config\_word()}'
within the kernel's PCI core that accesses PCI device configuration
space. As an exported symbol, it's available for external modules
to call. \footnote{./drivers/pci/access.c: \texttt{pci\_bus\_read\_config\_word()}}

\shadowbox{\begin{minipage}[t]{\mpwidth}
\begin{lstlisting}
int pci_bus_read_config_word(struct pci_bus *bus, unsigned int devfn, int pos, type *value)

EXPORT_SYMBOL(pci_bus_read_config_word)

0x1aa4c8bf pci_bus_read_config_word vmlinux EXPORT_SYMBOL
\end{lstlisting}
\end{minipage}}

From its function prototype, the '\texttt{int}' return type, its name,
and all its arguments, must remain preserved, which is somewhat self-evident.
Much more subtle, however, is its '\texttt{bus}' argument which is
a pointer to a '\texttt{struct pci\_bus}'. \footnote{\texttt{\label{fn:c1f6}include/linux/pci.h: struct pci\_bus \{ …
\}}} If, as the result of an upstream patch series incorporation, the
type of any member of '\texttt{struct pci\_bus}', including members
of structures within '\texttt{struct pci\_bus}', were to change --
including adding to, removing from, or changing the name or ordering
of any of its members -- then the '\texttt{pci\_bus\_read\_config\_word()}'
symbol's versioning checksum would correspondingly change.

Module versioning does not, nor does the Linux kernel, provide a \emph{stable
kernel interface} (a.k.a., a \emph{binary kernel interface}). \footnote{\href{https://www.kernel.org/doc/html/v4.12/process/stable-api-nonsense.html}{https://www.kernel.org/doc/html/v4.12/process/stable-api-nonsense.html}}
The KMI is solely a module to kernel consistency checking mechanism
for validating external kernel modules with respect to the kernels
they are attempting to link (attach) to, and run against.

\chapter{Technical Details: RHEL's kABI}

Throughout successive minor version update cycles of a RHEL release,
partners and customers request bug fixes, and additional features,
be applied to the kernel. Fulfilling these requests often requires
the removal of a function, changes to a function's return type or
parameters, or changes to kernel data structures (adding, removing,
and renaming members). These types of changes recurrently alter the
CRC versioning information of globally exported symbols; negating
the ability to continue running a previously runnable, external, module
upon updated kernels.

Like Linux's KMI, RHEL's kABI focuses on exported symbols, utilizing
the same underlying '\texttt{genksyms}' based symbol versioning information
in its checking mechanisms. So, while they are inter-related (any
symbol changes affects both kABI and KMI), KMI is concerned with after-the-fact
module to kernel compatibility checking, whereas RHEL's kABI is concerned
with before-the-fact preclusion of kernel to module incompatibilities.
kABI's role is to maintain a stable kernel interface, specifically
in order that RHEL's partner's external modules continue to operate
along minor kernel updates. \footnote{\label{fn:c3f1}Within a specific RHEL major release (the X in X.Y;
i.e., RHEL 7.5), an external module that was compiled against a specific
minor (Y) version should continue to operate on all subsequent (Y+;
i.e., RHEL 7.6, ...), but not necessarily previous (-Y; i.e., ...,
RHEL 7.4), minor versions.}

RHEL's kernel build related processes have been augmented with a set
of additional rules and scripts to implement kernel ABI stability.
RHEL's implementation utilizes the kernel's existing \texttt{\_\_GENKSYMS\_\_}
directive to dynamically enable, and disable, checking. Additionally,
with respect to partners, there are user-level enforcement checking
utilities ('\texttt{ksc(1)}'), all of which act in concert to enact
Red Hat's kABI commitment. \footnote{kABI policy strictly requires a module is compiled with a toolchain
(kernel headers, '\texttt{gcc}', '\texttt{glibc}', etc.) from a single
RHEL OS release. Modules compiled with a mix of toolchain components
are not supported.}

\section{What is Protected}

RHEL's kABI commitment does not cover all exported symbols but rather
is limited to, and thus only concerned with, the subset of global
symbols that partners have requested be preserved, and that Red Hat's
kernel development engineers have reviewed, and explicitly agreed
to protect. Such symbols become \emph{white-listed }{[}see: \nameref{chap:kABI-Lists}{]},
which is one of the components used in implementing RHEL's kABI compliance.

Agreeing to partners' external symbol requests has significant consequences.
More acutely, as a direct result of partners and customers developing
and compiling their external modules against a release's inception
(i.e., X.0), kABI's impacts, and restrictions, begin with its origin,
and persist throughout its entire lifecycle.

While RHEL's kABI is not limited to kernel data structures -- like
KMI, it focuses on kernel external symbols and their entire prototype's
makeup -- they are of particular interest with respect to kABI also.
Data structures have all the same impacts to kABI as they do to KMI:
ramifications can be very subtle, often outright hidden, yet potentially
encompass a very large scope.

Continuing with the example, '\texttt{pci\_bus\_read\_config\_word()}'
is a kernel exported symbol that is white-listed, thus considered
protected with respect to kABI.

\texttt{\footnotesize{}0x1aa4c8bf pci\_bus\_read\_config\_word vmlinux
EXPORT\_SYMBOL}\texttt{ }\footnote{RHEL7: ./redhat/kabi/kabi-module/kabi\_\{ppc64,s390x,ppc64le,x86\_64\}/pci\_bus\_read\_config\_word}

The kernel's '\texttt{pci\_dev}' data structure, via its indirect
correlation from the '\texttt{bus}' argument ('\texttt{struct pci\_dev}'
contains members of type '\texttt{struct pci\_bus}') of '\texttt{pci\_bus\_read\_config\_word()}',
also becomes effectively kABI protected. \footnote{./include/linux/pci.h: \texttt{struct pci\_dev \{ … \}}}
So: ``What does that imply with respect to kABI?''

Consider upstream commit \texttt{31c2b8153c ``PCI: Add support for
DMA alias quirks''}. As a core aspect of that commit, a new '\texttt{dma\_alias\_devfn}'
member was added to '\texttt{struct pci\_dev}'.

\shadowbox{\begin{minipage}[t]{\mpwidth}
\texttt{}
\begin{lstlisting}
diff git a/include/linux/pci.h b/include/linux/pci.h
index 545903df00dc..9d4035c276f4 100644
--- a/include/linux/pci.h
+++ b/include/linux/pci.h
@@ -268,6 +270,7 @@ struct pci_dev {
        u8              rom_base_reg;   /* which config register controls the ROM */
        u8              pin;            /* which interrupt pin this device uses */
        u16             pcie_flags_reg; /* cached PCIe Capabilities Register */
+       u8              dma_alias_devfn;/* devfn of DMA alias, if any */
 
        struct pci_driver *driver;      /* which driver has allocated this device */
        u64             dma_mask;       /* Mask of the bits of bus address this
\end{lstlisting}
\end{minipage}}

Prior to commit \texttt{31c2b8153c}, '\texttt{struct pci\_dev}' consumed
N bytes of memory, and the '\texttt{driver}' member was at some known
offset from the structure's beginning - assume 0x148. After the commit,
'\texttt{struct pci\_dev}' now consumes N + 1 bytes of memory, and
the pre-existing 'driver' member no longer resides at the same 0x148
byte offset location in memory that backs the structure (all '\texttt{struct
pci\_dev}' members at, or past, the '\texttt{driver}' member will
have been shifted to new, larger, offset locations).

Now consider a partner's external module which was compiled against
RHEL's 7.0 kernel, and relies upon (i.e., utilizes) the kernel's '\texttt{pci\_bus\_read\_config\_word()}'
external symbol. If upstream commit \texttt{31c2b8153c} were back-ported
into RHEL in a straightforward manner, say during 7.1's kernel development
phase, changes to the kernel's version of '\texttt{struct pci\_dev}'
would affect the module in two ways.

\begin{tabular}[t]{|c|c|c|c|}
\hline 
RHEL 7.0 kernel & Offset & RHEL 7.1 kernel & Offset\tabularnewline
\hline 
\hline 
\begin{lstlisting}
struct pci_dev {
  ...
  u16 pcie_flags_reg;
  struct pci_driver *driver;
  u64               dma_mask;
  ...
  ...
\end{lstlisting}
&
\begin{lstlisting}
...
...
0x146
0x148
0x150
...
...
\end{lstlisting}
&
\begin{lstlisting}
struct pci_dev {
  ...
  u16 pcie_flags_reg;
  u8                dma_alias_devfn;
  struct pci_driver *driver;
  u64               dma_mask;
  ...
\end{lstlisting}
&
\begin{lstlisting}
...
...
0x146
0x148
0x150
0x158
...
\end{lstlisting}
\tabularnewline
\hline 
\end{tabular}

First, any module's references to the kernel's structure's members
up to, but not including, the '\texttt{driver}' member would be fine,
as none of their offsets would have changed. However, unbeknownst
to the external module, a new '\texttt{dma\_alias\_devfn}' member
now resides within the kernel where the '\texttt{driver}' member used
to in the memory backing '\texttt{struct pci\_dev}'. So, in effect,
all member references by modules of the '\texttt{driver}' member onward
will dereference memory that no longer corresponds to what the module
expects. Second, since the structure is effectively larger now, if
the module accesses memory immediately after an automatic allocation
of the structure (i.e., a routine's local variable of type '\texttt{struct
pci\_dev}' within the module's source file) in any manner, those memory
dereferences also will not correspond to what the module is expecting.
Both conditions -- member dereferences at, or past, the newly added
member, and the structure's overall size -- would cause external
modules to dereference memory now backing other content resulting
in silent data corruption at best, and much more likely, outright
failure (i.e., a kernel panic), as a result of the kernel's kABI commitment
regarding '\texttt{pci\_bus\_read\_config\_word()}' having been broken.

The broken kABI commitment is not limited to the '\texttt{pci\_bus\_read\_config\_word()}'
symbol. The change to '\texttt{struct pci\_dev}' would correspondingly
cause all partners external modules that utilized any external symbols
with correlations to '\texttt{struct pci\_dev}' (in any capacity,
directly, or indirectly) to no longer successfully link and run against
such an updated kernel.

So, not only are kABI protected external symbols effectively locked
down with respect to any changes -- \emph{API permanence} -- all
data structures correlated with such symbols also become entangled
and are effectively locked down with respect to any changes, including
member additions, deletions, and re-naming or re-arrangement - \emph{kernel
data structure invariance}. In general: kABI protected external symbols,
and all kernel data structures that are correlated with such symbols,
are effectively locked down with respect to any changes, starting
with the symbol becoming protected (which may occur with the release's
origin), and persists throughout the release's entire lifecycle. \footnote{\label{fn:c3f5}Encumbered structures become locked down upon their
associated symbols becoming kABI protected, which may start occurring
as early as the origin release's Beta 6 snapshot, or within any subsequent
minor kernel update cycle thereafter.}

Due to the restrictive constraints that kABI imposes onto RHEL, Red
Hat does not automatically guarantee 100\% compliance of symbol protection
requests received. Some requested symbols are purposely rejected,
to allow the symbols to change in future minor releases in order to
minimize the encumbrances of integrating new, or maturing, technologies
from upstream. In such cases, partners may have to recompile their
modules for every minor update kernel. \footnote{Conversations with, and surveys of, key partners indicate that most
recompile their modules on every minor version update as a standard
practice.}

\section{Kernel Development Impacts}

While not all types of kABI induced ramifications and repercussions
could ever be covered, reviewing and understanding some key examples
from past experiences should help developers gather enough knowledge
with which to tackle future kABI breakages they encounter on their
own. There are cases of frequently occurring issues that are naturally
coincident with, and known to be vulnerable to, kABI and kernel development.
The most common can be roughly categorized into general areas susceptible
to kABI and include: exported symbols, data structures, header files,
static inline functions, enumerations, and the \texttt{/proc} (procfs)
and \texttt{/sys} (sysfs) filesystems.

RHEL kernel developers must be aware of the conditions that may introduce
kABI failures when developing, and back-porting, patches. This is
mostly a trial and error process at first but, with time, and experience,
there are general situations developers become familiar with, having
learned that such types of changes are particularly susceptible to
introducing kABI failures.

Within the classes outlined next are references to kABI case studies
presented in Appendix B ``\nameref{chap:kABI-Case-Studies}`` -
providing specific examples of methods developed for solving specific
kABI failure types. It is recommended to read the class' content first
to gain an overall understanding, and context, and then reference
the noted, class specific, case studies for further details.

\section{What is Impacted}

\subsection{Exported Symbols}

An exported symbol's return type can be augmented with '\texttt{const}'
(see: \texttt{RH\_KABI\_CONST()}). Other than that, changing an exported
symbol's prototype in any way -- its name, return type, and parameters,
including their ordering -- is disallowed if it's white-listed: API
permanence.

\subsection{Data Structures}

Kernel data structure changes are by far the most common kABI breakage
type. \footnote{These types of breakages are also hidden upon a successful compilation.}
Data structures are not symbols themselves, but become entangled with
a symbol due to their direct, and indirect, correlations. Direct correlations
are straightforward; the data structure is used as the symbol's return
type, or within the implementation by way of a parameter. Indirect
correlations, where data structures include other data structures,
are much more concealed. Such entangled data structure sets can become
potentially large in scope, making them very subtle, and thus extremely
hard, to associate as the specific cause of a breakage.

In general, developers cannot change kernel data structures correlated
with white-listed symbols unless some type of kABI-specific extension
mechanism has been pro-actively put into place. All kernel data structures
that are correlated with kABI protected symbols, are effectively locked
down with respect to any changes -- member additions, deletions,
re-naming, and re-arrangement -- starting with the symbol becoming
protected, and persisting throughout its entire lifecycle: kernel
data structure invariance. Otherwise, if data structures were altered
going forward, an external module built against a prior kernel would
cease to link successfully due to Red Hat not having maintained its
kABI commitment. This obviously severely impacts kernel development,
especially when \emph{foundational} data structures -- structures
that make up the core of some kernel subsystem's basis -- have become
encumbered with a white-listed symbol, sometimes to the extent that
no known recourse is available

Upstream data structure changes typically fall into one of three categories:
adding new members, deleting existing members, and re-naming or re-arranging
members. Deletions, and re-arrangements are fairly straightforward:
don't enact such changes as they are prohibited, and let RHEL's kernel
source vary from upstream in these specific cases. Additions of new
members are much more problematic as new features depend upon such.
Fortunately, some fairly inventive solutions have been developed over
the years to help with these specific instances.

There is a technique, commonly referred to as an ``extension'',
that can be used for adding new members, but, such RHEL specific extensions
to existing kernel structures must be put into place prior to, or
coincident with, the correlated symbol becoming protected. This assures
that any external modules will always have been compiled against a
kernel that already has the extensions, thus relevant symbol version
checksums will not have changed as far as any external modules are
concerned.

While kernel data structures may be able to be extended there is implicit
factor that has to be considered when developing an extension scheme.
Extending kernel data structures correlated with protected symbols
is all about enabling new feature content developed upstream, via
back-ports into RHEL, without breaking RHEL's kABI commitment. kABI
related extension mechanisms enable this, specifically preparing for
new members to be added to a structure encumbered with one or more
protected symbols which can take place within any minor update's development
cycle. As a result, from the kernel's perspective: protection encumbered
data structures with extension mechanisms enabled pro-actively may
change size (growing only) during any minor update cycle. However,
external modules that were compiled against earlier kernel versions,
and thus would not know about such structure growth (``kernel/module
variance''), must continue to link (attach) and run upon newer kernels
-- RHEL's kABI commitment. As such, the kernel's data structure growth
has to be enacted in a way that accounts for kernel/module variance.

The implication is that a symbol may become protected, and its encumbered
data structures thus locked down, with the release's origin (i.e.,
RHEL X.0). In fact, this is fairly common occurrence for the kernel's
foundational data structures as they are the most likely to become
encumbered with symbols becoming protected. Therefore, pro-actively
implementing extension members for key kernel data structures during
the release's origin is a prudent action that subsystem maintainers
should strongly consider. Otherwise, there likely will be no recourse
available: Adding extensions is a once per release event; living with
kABI is day to day.

For examples, and details, of extension structures along with one
other method related to ``holes'', see the kABI Case Studies related
to kernel data structures:
\begin{itemize}
\item \fullref{sec:Case-Study:Pro-actively-Extending}
\item \fullref{sec:Case-Study:Adding-Members-Centrally}
\item \fullref{sec:Case-Study:Adding-Members-Non-Centrally}
\item \fullref{sec:Case-Study:Exploting-Holes}
\item \fullref{sec:Case-Study:Kernel-Header-Files}
\end{itemize}
The extension implementation developed and used within the kernel's
networking subsystem is considered the canonical model to follow currently
{[}see: \fullref{sec:Case-Study:Pro-actively-Extending},
and \nameref{chap:kABI-Case-Studies}{]}.

Kernel data structures not encumbered with any white-listed symbols
are not protected by any kABI commitments.

\subsection{Header Files}

Including header files may change the parsing of structures - a source,
or header, file that had an indirect forward reference to a structure
is changed to now include another header which, via the inclusion,
replaces the indirect forward reference with the full declaration
of the structure (i.e., previously opaque symbols become transparent,
as the pre-processor now knows more about them). These types of occurrences
can result in a ``\emph{false-positive}'' kABI breakage, and can
be worked around by the use of the \texttt{\_\_GENKSYMS\_\_} pre-processing
directive.

kABI Case Studies related to kernel header files:
\begin{itemize}
\item \fullref{sec:Case-Study:Kernel-Header-Files}
\end{itemize}

\subsection{Static Inline Functions}

Developers should be aware that modifying public inline functions
might cause problems with external driver modules. Care should be
taken to minimize those changes and when possible use exportable functions
instead.

\subsection{Enumerations}

There are currently no hard requirements to protect these interfaces;
however, changes to enumeration ordering can affect external modules.
RHEL's current kABI checking capabilities may not catch simple enumeration
related changes that indeed affect modules, leading to \emph{silent
}kABI\emph{ failures}.

Enumerations can also be inherently problematic in nature as the C
Standard says un-assigned enumeration lists can have any value assigned
to each member of the list for any given compilation. Enumeration
lists that explicitly assign values avoid this issue, and converting
un-assigned lists to explicitly assigned lists can be used to obviate
future kABI concerns. Developers should self-police and preserve them
on a case-by-case basis.

kABI Case Studies related to enumerations:
\begin{itemize}
\item \fullref{sec:Case-Study:Enumerations}
\end{itemize}
For recurring situations where solutions have been developed over
the years that solve, or work around, kABI failures, case studies
have been written and are presented in Appendix B: ``kABI Case Studies:
Methods of Working Around kABI's Impacts''. Each study outlines the
case's general type, specifics involved, techniques used, explanations
as to why the solution works, and any known caveats to be cognizant
of.

Note however, there are situations in which developers encounter circumstances
where kABI precludes back-porting content from upstream and there
is no known recourse.

\section{What is Not Impacted}

\subsection{procfs / sysfs}

There are currently no hard requirements to protect these interfaces;
however, upstream generally disallows removals or non-compatible changes
(additions are allowed when justified). Developers will occasionally
self-police and preserve them on a case-by-case basis.

\subsection{What is \texttt{\_\_GENKSYMS\_\_} and its Abuse}

While \texttt{\_\_GENKSYMS\_\_} is not one of the common cases listed,
it is a key mechanism that must be used in effecting solutions in
all cases (i.e., most cases use \texttt{\_\_GENKSYMS\_\_} in some
capacity). Working around kABI is typically a two-pronged process.
First, a strategy that does not logically alter a kernel symbol's
ABI has to be developed (assuming one can be crafted, or even exists).
With a logical solution implemented, the second part of masking the
physical (i.e., code) changes from both KMI's, and kABI's, symbol
versioning based checking mechanisms, must be enacted. This involves
using the kernel's same build related pre-processing directive --
\texttt{\_\_GENKSYMS\_\_}.

kABI workarounds must effectively mask the specific changes that would
break kABI. To do such, RHEL implementations used to use a common
idiom of surrounding changes via \_\_GENKSYMS\_\_. \footnote{RHEL7: Commit \texttt{4f994b1e3fc} “PCI: Add support for DMA alias
quirks”} \footnote{\label{fn:s3f9}Note the new member is added to a RHEL specific “extension”
structure, not '\texttt{struct pci\_dev}' {[}see: Appendix AB.1 ,
B.3, and B.4{]}.}

\shadowbox{\begin{minipage}[t]{\mpwidth}
\begin{lstlisting}
@@ -379,6 +381,7 @@ struct pci_dev {
 */
 struct pci_dev_rh {
 #ifndef __GENKSYMS__
+        u8 dma_alias_devfn;/* devfn of DMA alias, if any */
 #endif
 };
\end{lstlisting}
\end{minipage}}

Over time, this idiom was becoming overused, leading to some negative
outcomes -- from the \emph{commit message} of RHEL7 commit \texttt{023ac646f29}
``kabi: Remove GENKSYMS usage through the code'':

\begin{lstlisting}[]
"A conversation started about the usage of GENKSYMS on a 7.0 kernel.

Normally this is not needed because kabi is not frozen until _after_ a .0 release is GA.

However folks were using this wrapper to clearly indicate which chunks of code were being
reserved for Red Hat purposes.

After thinking about this and discussing it with folks, we realized this is not a good idea.
Two reasons have stuck out as to why.

First, the idea was that hiding the symbols in the GENKSYSMS wrapper, we hide the true checksum
of the symbols. We should try to encompass the true checksum for GA. On the flip side,
removing the macro and doing things properly for 7.1 would entail:

        struct foo {
                int bar;
        #ifndef __GENKSYMS__
                int original;
        #else
                int new;
        #endif
        }

which leads to ugly and hard to read code. I agreed to create some macros in 7.1 that would
hide the ugliness, yet make it obvious what is going on.

Second, and this is more a social problem, most folks have come to assume that anything inside
__GENKSYMS__ is free to be modified/added. But in this particular example, this is _not_ the
case. So to prevent the confusion, we should avoid this usage for reserving memory.

As a result, I have removed all uses I believe were not correct. There should be no downside as
this only affects how genksyms pre-processes the symbols and generates checksums. The only
noticeable change will be new kabi checksums for a bunch of functions.

I did leave in a few usages, namely the PCI ones. I believe this is a correct usage as it
properly sets the stage for future additions to the reserved struct."
\end{lstlisting}

As a result of the actions behind the context of RHEL7 commit \texttt{023ac646f29}
(too many kABI-naive engineers overusing, and abusing, \texttt{\_\_GENKSYMS\_\_};
anything inside \texttt{\_\_GENKSYSMS\_\_} is not free to be modified/added),
RHEL specific kABI macros were developed, and incorporated, during
7.1's minor update development cycle {[}see: Appendix C \ref{sec:include/linux/rh_kabi.h}
\nameref{sec:include/linux/rh_kabi.h}{]}. \footnote{\label{fn:c3f10}RHEL7: \texttt{include/linux/rh\_kabi.h} (via commit
\texttt{22f7116921c} “kabi: Add generic kabi macros to use for kabi
workarounds”).} The kABI macros were introduced for use in kABI related development
efforts that effectively do the same thing but in a manner identical
to the kernel's build processing - using macros that expose changes
when disabled yet mask changes (essentially turning them into a no-op,
or \texttt{<null>}, clause), when enabled.

\shadowbox{\begin{minipage}[t]{\mpwidth}
\begin{lstlisting}
#ifdef __GENKSYMS__
...
#        define RH_KABI_EXTEND(_new)
...
#else
...
#        define RH_KABI_EXTEND(_new) _new;
...
#endif /* __GENKSYMS__ */
\end{lstlisting}
\end{minipage}}

As should be evident, blindly using RHEL's '\texttt{RH\_KABI\_EXTEND()}'
macros in an attempt to workaround kABI is not sufficient -- a common
mistake made by RHEL kernel developers not fully cognizant of kABI
-- as the core ABI issues still exist (the ``first'' step mentioned
above is needed to solve the ``logical'' aspects). In fact, attempting
to do such just ``hides'' the breakage from the kABI enforcement
tools' detecting such, a much worse problem.

All of the kABI case studies outlined in ``\nameref{chap:kABI-Case-Studies}''
utilize common \texttt{\_\_GENKSYMS\_\_} based idioms, either directly,
or indirectly via RHEL's kABI macros, as one aspect in an overall
solution.

\section{Verification}

When building a RHEL kernel, either from a Source RPM (SRPM) or via
'\texttt{rh-brew}', a kABI checking script, '\texttt{check-kabi}',
is automatically run as part of the build process. \footnote{\texttt{redhat/kabi/check-kabi}}
The script attempts to check for any kABI breakages introduced by
changes to the kernel. The checking is constrained to RHEL's white-listed
symbols only; it does not check the kernel's non white-listed exported
symbols

kABI protection applies to all supported architectures. Hence building
via '\texttt{rh-brew}' is the recommended method, as it will run '\texttt{check-kabi}'
on all architectures. If '\texttt{check-kabi}' is executed manually,
only one specific architecture is checked. When building via the '\texttt{rpmbuild}'
process, the SPEC file sets the '\texttt{with\_kabichk}' flag --
\texttt{redhat/rpm/SPECS/kernel.spec::\%define with\_kabichk} --
enacting the exact same '\texttt{check-kabi}' based checking.

To manually check if there is going to be a kABI issue, prepare by
first enacting the patch series' changes, and make the kernel:

\shadowbox{\begin{minipage}[t]{\mpwidth}
\begin{lstlisting}[language=sh]
$ make rh-configs-prep
$ cp configs/kernel-2.6.32-x86_64.config .config
$ make oldconfig
$ make -j16
\end{lstlisting}
\end{minipage}}

With the kernel built, manually run '\texttt{check-kabi}' to check
for kABI failures:

\shadowbox{\begin{minipage}[t]{\mpwidth}
\begin{lstlisting}[language=sh]
$ redhat/kabi/check-kabi -h /* help */
$ redhat/kabi/check-kabi -k redhat/kabi/Module.kabi_x86_64 -s Module.symvers
\end{lstlisting}
\end{minipage}}

If there is no output, then the build was successful and there were
no kABI related failures. Otherwise, '\texttt{check-kabi}' outputs
a list of broken symbols.

\section{Debugging Breakages}

Removal or change to white-listed symbols is considered a serious
offense as partners and customers are actively using these symbols.
Changing a white-listed symbol's checksum would cause their modules
to fail loading in future minor updates, negating Red Hat's commitment.
Red Hat has removed white-listed symbols in the past, but only after
contacting partners and securing knowledge they were not using those
symbols.

Developers will, on occasion, include a patch that incidentally modifies
a white-listed symbol. This will lead to the checksum changing and
is considered ``\emph{breaking kABI}''. In such cases, developers
are expected to modify their patches to accommodate these symbols
and patches will be rejected unless they are revised to work around
such breakages.

\subsection{via 'check-kabi'}

Inevitability, every RHEL kernel developer will encounter the case
where a patch series is applied and building a kernel now fails due
to kABI (specifically, \texttt{'check-kabi}' fails). How can one determine
which part of a patch caused the breakage, especially if the broken
symbol seems to have nothing to do with the applied patch series?

Novell added some code a while ago that saves the 'genksyms' versioning
symbol(s) backtrace when compiling {*}.o files. Using this backtrace
one can hone in and determine where the kABI broke by comparing the
old backtrace with the new one. \footnote{RHEL6: commit \texttt{409236c741a} Makefile.build: make KBUILD\_SYMTYPES
work again}
\begin{itemize}
\item Make sure you have a clean master branch in git and another branch
with your changes (it is just easier this way); 
\item Make sure you have your .config setup; 
\item Pick a symbol that brew said broke with respect to kABI, call it '\texttt{foo}'. 
\end{itemize}
\shadowbox{\begin{minipage}[t]{\mpwidth}
\begin{lstlisting}[language=sh]
$ git checkout master                    # Need the original kABI cksum
$ git grep foo | grep EXPORT_SYMBOL      # Find the file that exports foo
$ make file.symtypes                     # Generate backtrace
$ cp file.symtypes file.symref           # Save the backtrace
$ git checkout branch_with_change        # Switch to patched branch
$ KBUILD_PRESERVE=1 make file.symtypes   # Generate and compare new backtrace
\end{lstlisting}
\end{minipage}}

By way of concrete example, assume some changes have broken kABI and
'\texttt{check-kabi}' indicates that\newline the
'\texttt{ieee80211\_restart\_hw}' symbol now fails:

\shadowbox{\begin{minipage}[t]{\mpwidth}
\begin{lstlisting}[language=sh]
$ git grep <target_kABI_failure_symbol_from_check-kabi> | grep EXPORT
net/mac80211/main.c:EXPORT_SYMBOL(ieee80211_restart_hw);
\end{lstlisting}
\end{minipage}}

According to '\texttt{check-kabi}', there looks to be an issue with
the '\texttt{ieee802111\_restart\_hw}' symbol but in reviewing the
patch series' content there are no changes to that function's body
of code.

Continuing with the approach outlined above:

\shadowbox{\begin{minipage}[t]{\mpwidth}
\begin{lstlisting}[language=sh]
$ make net/mac80211/main.symtypes    # Note: strip the .c from source filename
\end{lstlisting}
\end{minipage}}

This should fail and print out the structure that caused the breakage:

\shadowbox{\begin{minipage}[t]{\mpwidth}
\begin{lstlisting}[language=sh,basicstyle=\scriptsize\ttfamily]
$ KBUILD_PRESERVE=1 make init/version.symtypes
...
init/version.c:37: error: init_uts_ns: modversion changed because of changes in struct uts_namespace
...
\end{lstlisting}
\end{minipage}}

So the issue isn't within the '\texttt{ieee80211\_restart\_hw}' symbol's
function, but rather something indirectly involving '\texttt{struct
uts\_namespace}'. Reviewing the patch series again, with this new
information, one now sees that it indeed added a new member to \texttt{include/linux/utsname.h::'struct
uts\_namespace'} which is the root of the kABI failure:

\shadowbox{\begin{minipage}[t]{\mpwidth}
\begin{lstlisting}
-@ -40,6 +40,7 @@ struct new_utsname {
 struct uts_namespace {
         struct kref kref;
         struct new_utsname name;
+        struct dummy;
};
extern struct uts_namespace init_uts_ns;
\end{lstlisting}
\end{minipage}}

Most kABI breakages usually center on a single change, but sometimes
more than one change breaks kABI. One can either perform the above
steps for each symbol that broke kABI or just recompile and see what
was missed. Recompiling is probably easiest but may not be enough.

\subsection{via 'kabi-dw'}

'\texttt{kabi-dw}' is a new tool which allows checking for kABI changes
based on changes in DWARF debugging data. In short it loads the type
information from DWARF, from vmlinux and \emph{all kernel modules},
and dumps these types into a set of text files starting with a given
list of exported symbols and descending to all types reachable from
these symbols.

Consider the following example where we are interested only in '\texttt{pci\_bus\_read\_config\_word()}'
symbol. First we'd build the kernel tree with debug info enabled (which
is on by default in RHEL kernels):

\shadowbox{\begin{minipage}[t]{\mpwidth}
\begin{lstlisting}[language=sh]
$ make rh-configs
$ cp redhat/configs/kernel-3.10.0-$(uname -p).config .config
$ make && make modules
\end{lstlisting}
\end{minipage}}

Then we'd generate the list of symbols we're interested in:

\shadowbox{\begin{minipage}[t]{\mpwidth}
\begin{lstlisting}[language=sh]
$ echo 'pci_bus_read_config_word' > symbols
\end{lstlisting}
\end{minipage}}

We build the kabi-dw tool:

\shadowbox{\begin{minipage}[t]{\mpwidth}
\begin{lstlisting}[language=sh]
$ make rh-kabi-dw
\end{lstlisting}
\end{minipage}}

Finally we generate the files for all types reachable from our selected
symbol:

\shadowbox{\begin{minipage}[t]{\mpwidth}
\begin{lstlisting}[language=sh]
$ redhat/kabi-dwarf/kabi-dw/kabi-dw generate -s symbols -o output .
\end{lstlisting}
\end{minipage}}

In the 'output' directory we can find the text file for function '\texttt{pci\_bus\_read\_config\_word}':

\shadowbox{\begin{minipage}[t]{\mpwidth}
\begin{lstlisting}[language=sh]
$ ls -l output/func-pci_bus_read_config_word.txt
-rw-r-r-. 1 root root 175 Apr 17 16:38 output/func-pci_bus_read_config_word.txt
\end{lstlisting}
\end{minipage}}

Because the '\texttt{bus}' argument of '\texttt{pci\_bus\_read\_config\_word()}'
is of type '\texttt{struct pci\_bus {*}}' there will also be a file
for '\texttt{struct pci\_bus}':

\shadowbox{\begin{minipage}[t]{\mpwidth}
\begin{lstlisting}[language=sh]
$ ls -l output/struct-pci_bus.txt
-rw-r-r-. 1 root root 1051 Apr 17 16:38 output/struct-pci_bus.txt
\end{lstlisting}
\end{minipage}}

Similarly the tool descends to all the reachable types that are defined
in the DWARF debug data.

It is expected that the list of symbols provided to start the dump
of data will be the kABI white-list. There is Makefile target which
simplifies whole process with providing the current whitelist as the
input and storing the output under \texttt{redhat/kabi-dwarf/base/\$(uname
-p)}:

\shadowbox{\begin{minipage}[t]{\mpwidth}
\begin{lstlisting}[language=sh]
$ make rh-kabi-dw-base
\end{lstlisting}
\end{minipage}}

'\texttt{kabi-dw}' tool can also be used for comparing two datasets.
Typical usage would then include generating baseline dataset (aka.
last offical RHEL build), re-building the kernel with your changes,
generating new dataset and comparing them with \texttt{kabi-dw}:

\shadowbox{\begin{minipage}[t]{\mpwidth}
\begin{lstlisting}[language=sh]
$ redhat/kabi-dwarf/kabi-dw/kabi-dw compare output1 output2
\end{lstlisting}
\end{minipage}}

This can again be simplified with:

\shadowbox{\begin{minipage}[t]{\mpwidth}
\begin{lstlisting}[language=sh]
$ make rh-kabi-dw-base
# ... apply your patch
$ make && make modules
$ make rh-kabi-dw-check
\end{lstlisting}
\end{minipage}}

\subsection{Where can kabi-dw help?}

In general, the '\texttt{kabi-dw}' approach is similar to the '\texttt{make
<file>.symtypes}' approach described above. The main added value is:
\begin{itemize}
\item '\texttt{kabi-dw}' can ``see'' through all '\texttt{\_\_GENKSYMS\_\_}'
macros. That is, even if certain parts of the code are already masked
using '\texttt{\_\_GENKSYMS\_\_}' or some of the '\texttt{RH\_KABI\_{*}()}'
macros, these changes will still be reported in '\texttt{kabi-dw}'
output.
\item '\texttt{kabi-dw}' looks at the whole kernel tree, ie. \texttt{vmlinux}
+ \emph{all modules}, together. This is in sharp contrast with '\texttt{genksyms}'
which only looks at single C source file at a time. '\texttt{kabi-dw}'
can help to answer questions like if given structure is embedded inside
other structure elsewhere in the kernel.
\end{itemize}

\chapter{Conclusion}

Attempting to describe all the various ways that kABI impacts kernel
development is impractical. Developers will inevitability run into
new, unique, situations involving kABI. So while the content presented
just scratches the surface, hopefully, it will have been enough to
help developers in solving future kABI breakages that they encounter
on their own (when possible).

Anyone who has further questions related to kABI should send questions
to the \emph{core-kernel@redhat.com} mail list.

\section{Exercises}
\begin{enumerate}
\item Q: Explain what a Linux kernel ``exported symbol'' is. 
\item Q: What is KMI? 
\item Q: What is ``symbol versioning'' and what does it do? 
\item Q: What is kABI? 
\item Q: How is kABI different from KMI? 
\item Q: Does KMI catch all kABI breakages? Why? 
\item Q: Is it possible to break kABI in a way where KMI's symbol versioning
doesn't change and '\texttt{insmod}' still loads the module? Or, does
KMI symbol versioning causing modules to no longer load successfully
also catch all kABI breakages?  Answer: KMI should catch all breakages also but manual use (overuse)
of \texttt{\_\_GENKSYMS\_\_} related to kABI can hide changes causing
the association to no longer hold true. Additionally, \texttt{\_\_GENKSYMS\_\_}
will not catch enumeration changes unless the enumeration is hardwired
in an inline function (i.e., enumeration changes can silently break
kABI). 
\item Q: Can all kernel data structures be extended? Why not? 
\item Q: Is using a RHEL kABI macro such as '\texttt{RH\_KABI\_EXTEND()}'
sufficient to satisfy all requirements when dealing with a kABI breakage?
Why?\pagebreak{}
\end{enumerate}

\appendix

\chapter{kABI Lists\label{chap:kABI-Lists}}

kABI tracks global symbols through two types of lists: white, and
grey. These lists define which symbols are protected and which ones
are not:

\subsubsection{White}

Exported symbols that Red Hat has committed to not break throughout
the subsequent lifecycle of a RHEL major release make up the ``white-list''.
\footnote{'RHEL KABI Whitelist Process and Program', Red Hat, 11 August 2017.

\url{https://docs.google.com/document/d/1cEj4sYYacx16vHj3-41eEWqT-Wzq01VR7ydPcinIHQc/edit}}

Each minor RHEL release has a set of symbols constituting a white-list.
\footnote{White-listed symbols reside in:

kernel package - \texttt{/boot/symvers.\$(uname -r).gz}

kernel-devel package - \texttt{/usr/src/kernels/\$(uname -r)/Module.symvers}

kernel's source tree - \texttt{redhat/kabi/kabi-module/kabi\_\{ppc64,ppc64le,s390x,x86\_64\}/{*}}} Once a symbol has been added to the kABI white-list for a particular
release it will not be removed, nor is it allowed to change during
the major release's remaining lifecycle. New symbols may be added
to the kABI white-list in minor releases.

\subsubsection{Grey}

Exported symbols that Red Hat's partners have requested for review
and addition to the while-list but have not yet been explicitly accepted
nor rejected (e.g., they are in the internal review phase where Red
Hat's developers discuss whether or not to add them to the white-list).

There is a database server that maintains grey-listed entries for
each specific RH partner. This server also offers a monitoring service
that notifies partners if any of the symbols on their respective grey-list
changes.

Partners can request additions to RHEL's kABI white-list by opening
a RHEL Jira under the ``kernel / kabi-whitelists'' component.
Symbol requests are sent to the Red Hat Kernel development
team's Mail List (RHKML) for internal review. Symbols explicitly accepted
for protection are added to the kABI white-list at the end of the
next minor release's development phase. Symbols that were not explicitly
accepted are automatically added to the grey-list.

\chapter{kABI Case Studies: Methods of Working Around kABI's Impacts \label{chap:kABI-Case-Studies}}

Attempting to describe all the various ways that kABI impacts kernel
development is not practical. Nevertheless, there are sets of issues
that are recurrent, falling within somewhat common areas, scenarios,
or themes (i.e., classes of scenarios that can be roughly categorized).
For the recurrent issues within these classes, there have been some
fairly innovative methods developed over the years to work around
kABI's restrictions. So while not all types of issues could ever be
covered, going over and understanding some key examples from past
experiences should help developers gather enough knowledge with which
to tackle future kABI breakages they run into on their own.

This appendix outlines some of the more interesting, and common, past
experiences within the various classes via concrete examples. Each
study outlines the case's general scenario, specifics involved, techniques
used, explanations as to why the solution works, and any known caveats
to be cognizant of.

For each of the case studies, the referenced material ('\texttt{git}'
commits and source files) should be accessed and studied in an effort
to fully comprehend how, and why, the methods are successful in working
around kABI while maintaining an external module to continue to operate
across RHEL's successive minor update cycles despite kernel/module
variance.

\section{Case Study: Pro-actively Extending Core Kernel Data Structures \label{sec:Case-Study:Pro-actively-Extending}}

Upstream reference: N/A

RHEL7 reference: commit \texttt{c9be03bbc1c} \footnote{\label{fn:a1}{[}RHEL7 RHEL\_only PATCH v3 0/6{]} PCI: Add kABI shadow
structs for the kernel's PCI subsystem

\texttt{c9be03bbc1c} PCI: Create 'pci\_dev\_rh' KABI shadowing structure

\texttt{9599688156b} PCI: Convert 'pci\_dev' allocations and frees

\texttt{5abdc3d2dac} PCI: Convert specific 'pci\_dev' related constructs

\texttt{829d829c1b7} PCI: Create 'pci\_bus\_rh' KABI shadowing structure

\texttt{8af377d9592} PCI: Convert 'pci\_bus' allocations and frees

\texttt{610b1c95049} PCI: Create 'pci\_driver\_rh' KABI shadowing
structure}

This case study covers the pre-requisite setup that enables logically
adding new members to existing kernel data structures that are becoming,
or expected to become, encumbered with kABI-protected symbols. While
specific implementations have evolved over the years -- hash based
associations to shadow structures, directly associated opaque extension
structures, and pre-reserved additions with directly associated opaque
extension structures -- its central tenant -- adding a new data
structure that is directly associated to an existing data structure
unto which all future member additions will be added so as to not
effect any symbol versioning aspects of the original data structure
-- has not.

During RHEL7's initial kernel development phase -- 7.0 -- commit
\texttt{c9be03bbc1c} ``PCI: Create 'pci\_dev\_rh' KABI shadowing
structure'' was pro-actively put into place adding a pointer to an
opaque, RHEL kABI-specific, extension structure onto the end of '\texttt{struct
pci\_dev}'.\textsuperscript{\ref{fn:a1}} This pre-requisite step
allowed RHEL7 to maintain kABI with respect to '\texttt{pci\_dev}'
yet also accommodates logical upstream updates to such -- the addition
of new members only - via its '\texttt{pci\_dev\_rh}' extension structure.

\shadowbox{\begin{minipage}[t]{\mpwidth}
\begin{lstlisting}
diff -git a/include/linux/pci.h b/include/linux/pci.h
index e564932..45f1c16 100644
--- a/include/linux/pci.h
+++ b/include/linux/pci.h
@@ -364,6 +364,20 @@ struct pci_dev {
 #endif
         phys_addr_t rom; /* Phys addr of ROM if it's not from the BAR */
         size_t romlen; /* Length of ROM if it's not from the BAR */
+
+        /* Extension to accomodate future upstream changes to this struct
+         * yet maintain RHEL7 kABI.  For Red Hat internal use only!
+         */
+        struct pci_dev_rh  *pci_dev_rh;
+};
+
+/*
+ * RHEL7 specific 'struct pci_dev' shadow structure to help maintain kABI
+ * going forward.  This structure will never be under kABI restrictions.
+ */
+struct pci_dev_rh {
+#ifndef __GENKSYMS__
+#endif
 };

 static inline struct pci_dev *pci_physfn(struct pci_dev *dev)
\end{lstlisting}
\end{minipage}}

The RHEL specific extension member addition to existing kernel structures
-- in this instance, a pointer to an opaque extension structure --
must be put into place prior to, or coincident with, the correlated
symbol becoming protected. This assures that any external modules
will have always been compiled against a kernel that has the extension's
member included, prior to the symbol having become protected, and
thus relevant symbol version checksums will not have changed.

Once a symbol becomes protected, any encumbered data structures are
effectively locked down from any further changes for the remaining
life cycle of the release. Otherwise, if data structures were altered
going forward, an external module built against a prior kernel, would
cease to link successfully due to Red Hat not having maintained its
kABI commitment.

The implication is that a symbol may become protected, and its encumbered
data structures thus locked down, with the release's origin (i.e.,
RHEL X.0). In fact, this is fairly common occurrence for the kernel's
foundational data structures as they are the most likely to become
encumbered with symbols becoming protected. Therefore, pro-actively
implementing extension members for key kernel data structures during
the release's initial phase is a prudent action that subsystem maintainers
should strongly consider.

The subsequent step of allocating the extension structure itself can
be postponed until such time as needing to effectively add a new member
to the existing structure occurs {[}see: \ref{sec:Case-Study:Adding-Members-Centrally}
\nameref{sec:Case-Study:Adding-Members-Centrally} and \ref{sec:Case-Study:Adding-Members-Non-Centrally}
\nameref{sec:Case-Study:Adding-Members-Non-Centrally}{]}.

\subsection{Caveats}

There are some fairly strict requirements that the existing kernel
data structure predicted to need members added to at some future time
must meet {[}see: \fullref{sec:Case-Study:Adding-Members-Centrally}
and \fullref{sec:Case-Study:Adding-Members-Non-Centrally}{]}.

This method is not allowed after an encumbered data structure's symbol
has become protected. The extension member must be implemented prior
to becoming locked down, possibly pro-actively as part of the release's
origin. This implies that any kernel data structures that will need
members added to them at some future date should be predicted, and
extensions enacted, up front whenever possible, via ``experience''.

\section{Case Study: Adding Members to Centrally Allocated Kernel Data Structures
\label{sec:Case-Study:Adding-Members-Centrally}}

Upstream reference: commit \texttt{338c3149a22}

RHEL7 back-port ref: commit \texttt{8e7c115586d} \footnote{\texttt{8e7c115586d} PCI: Add support for multiple DMA aliases}

The case study involves adding new members to existing kernel data
structures. Its focus is the specific case where all allocations of
the existing structure needing augmented are performed by a single,
centralized, in-kernel-core routine. There are multiple aspects involved
to making this technique work without breaking kABI. This technique
requires a pre-requisite setup step {[}see: \nameref{sec:Case-Study:Pro-actively-Extending}{]}
that itself must be fully comprehended before proceeding.

During RHEL 7.3's development phase, upstream commit \texttt{338c3149a22}
``PCI: Add support for multiple DMA aliases'' was needed. The new
feature was dependent upon a new '\texttt{dma\_alias\_mask}' member
being added to '\texttt{struct pci\_dev}'.

\shadowbox{\begin{minipage}[t]{\mpwidth}
\begin{lstlisting}
diff -git a/include/linux/pci.h b/include/linux/pci.h
index 7e7019064437..5581d05f7833 100644
--- a/include/linux/pci.h
+++ b/include/linux/pci.h
@@ -273,7 +271,7 @@ struct pci_dev {
         u8              rom_base_reg;   /* which config register controls the ROM */
         u8              pin;            /* which interrupt pin this device uses */
         u16             pcie_flags_reg; /* cached PCIe Capabilities Register */
-        u8              dma_alias_devfn;/* devfn of DMA alias, if any */
+        unsigned long   *dma_alias_mask;/* mask of enabled devfn aliases */

         struct pci_driver *driver;      /* which driver has allocated this device */
         u64             dma_mask;       /* Mask of the bits of bus address this
\end{lstlisting}
\end{minipage}}

The '\texttt{pci\_dev}' data structure from the kernel's PCI subsystem
is foundational to its subsystem -- thus is altered by upstream changes
on a consistent basis.\textsuperscript{\ref{fn:c1f6}} Additionally,
being such a core data structure, there are numerous kABI-protected
symbols within PCI's subsystem that reference it, so any changes to
it would normally break kABI for many symbols. Fortunately, '\texttt{struct
pci\_dev}' meets very strict requirements that allows new members
to be logically added to it: it is strictly allocated from the kernel's
heap via a common function - '\texttt{pci\_alloc\_dev()}'. \footnote{\texttt{drivers/pci/probe.c::pci\_alloc\_dev()} (and \texttt{pci\_release\_dev()})}

If a data structure is strictly allocated from the kernel's heap (i.e.,
no static definitions of the structure are allowed, nor are automatic
allocations on the stack) then there is a technique that can be used
for adding new members to it without breaking kABI. \footnote{Strictly allocated from the kernel's heap implies: the data structure
was dynamically allocated, and references to it must be indirect (i.e.,
via a pointer).}

Intuitive, or experienced, developers will eventually ask: ``What
prevents an external driver module from allocating a '\texttt{struct
pci\_dev}' within itself, either statically (within the source file's
scope) or automatically (as an automatic variable within a routine;
thus resides on the routine's stack)?'' \footnote{The possible effects of such a scenario and how such breaks kABI are
covered in Appendix C: kABI: Canonical Method for Extending Kernel
Data Structures.} Nothing, however if one researches, via \texttt{'cscope(1)}' or some
similar tool, all 'struct pci\_dev' allocations within the kernel
and it's built-in modules, they will see that all allocations of '\texttt{pci\_dev}'
utilize '\texttt{pci\_alloc\_dev()}'. This is where kABI enters a
``grey area'': nothing prevents an external driver module from doing
such, but, it would be considered ill formed; improper programming
by a module is not considered a kABI breakage.

With the aforementioned setup step -- the extension mechanism from
commit \texttt{c9be03bbc1c} -- in place, the subsystem's common allocation
related functions next need to be augmented to include allocating
the additional extension structure along side its parent. While it
makes sense to do this coincident with the addition of the extension
structure in this particular case -- where allocations are handled
by a common in-kernel-core function within the subsystem -- it is
not absolutely necessary to perform this specific step until a specific
need arises as is covered in a different case study {[}see: \ref{sec:Case-Study:Adding-Members-Non-Centrally}
\nameref{sec:Case-Study:Adding-Members-Non-Centrally}{]}. This was
accomplished via RHEL7 commit \texttt{9599688156b} ``PCI: Convert
'pci\_dev' allocations and frees''.\textsuperscript{\ref{fn:a1}}

\shadowbox{\begin{minipage}[t]{\mpwidth}
\begin{lstlisting}
diff -git a/drivers/pci/probe.c b/drivers/pci/probe.c
index 38e403dddf6..24da7642d3c 100644
--- a/drivers/pci/probe.c
+++ b/drivers/pci/probe.c
@@ -1170,6 +1170,7 @@ static void pci_release_dev(struct device *dev)
         pci_release_of_node(pci_dev);
         pcibios_release_device(pci_dev);
         pci_bus_put(pci_dev->bus);
+        kfree(pci_dev->pci_dev_rh);
         kfree(pci_dev);
 }
 
@@ -1234,6 +1235,12 @@ struct pci_dev *pci_alloc_dev(struct pci_bus *bus)
         if (!dev)
                 return NULL;

+        dev->pci_dev_rh = kzalloc(sizeof(struct pci_dev_rh), GFP_KERNEL);
+        if (!dev->pci_dev_rh) {
+                kfree(dev);
+                return NULL;
+        }
+
         INIT_LIST_HEAD(&dev->bus_list);
         dev->dev.type = &pci_dev_type;
         dev->bus = pci_bus_get(bus);
@@ -1308,6 +1315,7 @@ static struct pci_dev *pci_scan_device(struct pci_bus *bus, int devfn)
 
         if (pci_setup_device(dev)) {
                 pci_bus_put(dev->bus);
+                kfree(dev->pci_dev_rh);
                 kfree(dev);
                 return NULL;
 }
\end{lstlisting}
\end{minipage}}

With an opaque, RHEL specific extension structure now in place, and
fully backed (allocated), future upstream changes to the structure
which would normally break kABI can now be brought into RHEL by placing
new members into the corresponding extension structure such as was
done in this case study's RHEL7 commit - \texttt{8e7c115586d} ``PCI:
Add support for multiple DMA aliases''.

\shadowbox{\begin{minipage}[t]{\mpwidth}
\begin{lstlisting}
@@ -398,7 +396,7 @@ struct pci_dev {
  * going forward.  This structure will never be under KABI restrictions.
  */
 struct pci_dev_rh {
-        RH_KABI_EXTEND(u8   dma_alias_devfn) /* devfn of DMA alias, ..
+        RH_KABI_EXTEND(unsigned long *dma_alias_mask) /* mask of ..
         RH_KABI_EXTEND(char *driver_override) /* Driver name to ..
 #ifdef CONFIG_PPC64
         RH_KABI_EXTEND(struct pci_dn *pci_data) /* PCI device node*/
\end{lstlisting}
\end{minipage}}

Any existing external module references to members of '\texttt{struct
pci\_dev}' will remain valid. Yes, there will effectively be new members
to '\texttt{struct pci\_dev}', placed into the extension structure;
however, external modules compiled against any earlier RHEL kernel
will not be aware of such additions and so will never attempt to access
the member or its backing memory.

With a solution in place that is understood not to logically break
existing external modules, the change would still end up altering
all symbols with associations to '\texttt{struct pci\_dev}'. The final
piece of the overall solution is to use RHEL's '\texttt{RH\_KABI\_EXTEND()}'
macro to circumvent \texttt{\_\_GENKSYMS\_\_} based checking from
detecting the change.\textsuperscript{\ref{fn:c3f10}}

\subsection{Caveats}

There are some fairly strict requirements that the existing kernel
data structure predicted to need members added to at some future time
must meet.

This method is not allowed after an encumbered data structure's symbol
has become protected. The extension member must be implemented prior
to becoming locked down, possibly pro-actively as part of the release's
origin.

One concerning caveat to this method is bringing in references to
any new members introduced in this fashion from upstream. As a direct
result of new members being added to RHEL's specific extension structure,
references to any such members will have to be adjusted to account
for the extra level of indirection incurred when back-porting (i.e.,
upstream references like 'pci\_dev-><member>' have to be converted
to '\texttt{pci\_dev->pci\_dev\_rh-><member>}').

Only the addition of new members is allowed. Member deletions, and
even re-arrangements, are prohibited.

\section{Case Study: Adding Members to Non-Centrally Allocated Kernel Data
Structures \label{sec:Case-Study:Adding-Members-Non-Centrally}}

Upstream reference: commit 2d984ad132a

RHEL7 back-port reference: commit bef916144ab \footnote{RHEL7: commit \texttt{bef916144ab} ``PM: add Red Hat dev\_pm\_info\_rh
struct''}

This case study is similar to \ref{sec:Case-Study:Adding-Members-Centrally}
\nameref{sec:Case-Study:Adding-Members-Centrally} in that it is concerned
with adding new members to an existing kernel data structure, in this
case '\texttt{struct device}'. \footnote{\texttt{include/linux/device.h::struct device \{ … \}}}What
differs, however, is that the existing kernel data structure does
not have a centralized allocation routine with which to collocate
the extension's allocation, introducing an additional consideration
that has to be taken into account, making things quite a bit more
complex, and worth specifically looking at.

The same, pre-requisite, associated extension structure setup is necessary
and took place during RHEL7's initial kernel development phase, via
commit \texttt{9f84482488d} ``device: Create 'device\_rh' KABI shadowing
structure'', which put a pointer to a RHEL specific kABI extension
structure, '\texttt{struct device\_rh}', onto the end of '\texttt{struct
device}'. \footnote{RHEL7: {[}PATCH 0/2{]} driver-core: Add KABI shadow structs to kernel's
driver subsys

\texttt{3ba6b481780} device: Create 'device\_driver\_rh' KABI shadowing
structure

\texttt{9f84482488d} device: Create 'device\_rh' KABI shadowing structure}

\shadowbox{\begin{minipage}[t]{\mpwidth}
\begin{lstlisting}
diff -git a/include/linux/device.h b/include/linux/device.h
index 19f4329b539..f8374e30669 100644
--- a/include/linux/device.h
+++ b/include/linux/device.h
@@ -754,6 +754,19 @@ struct device {

         bool                                offline_disabled:1;
         bool                                offline:1;
+
+        /* Extension to accomodate future upstream changes to this structure
+         * yet maintain RHEL7 KABI.  For Red Hat internal use only!
+         */
+        struct device_rh        *device_rh;
+};
+
+/* RHEL7 specific 'struct device_rh' shadow structure to help maintain KABI
+ * going forward.  This structure will never be under KABI restrictions.
+ */
+struct device_rh {
+#ifndef __GENKSYMS__
+#endif
 };
 
 static inline struct device *kobj_to_dev(struct kobject *kobj)
\end{lstlisting}
\end{minipage}}

The '\texttt{device}' structure, from the kernel's device driver subsystem,
underlies every device that runs on Linux. The sequence that device
drivers typically use with respect to getting a device initialized
-- \emph{device registration} -- is to first obtain an allocated
'device' structure, typically via inheritance from a device specific
subsystem, and then, at some later time, call '\texttt{device\_add()}'
which adds the device entry to the kernel with reference counting.
As such, '\texttt{struct device}' does not have a centralized function
for its allocation; allocations are distributed throughout the kernel.

Distributed allocations of '\texttt{device}' structures, typically
from within each of the kernel's specific device related subsystem's
themselves, introduces an additional consideration that must be taken
into account: ``Where is a good place to add the allocation of the
corresponding kABI extension structure?'' When the need to add a
member to RHEL's '\texttt{device}' structure finally arose -- during
the kernel development phase of RHEL 7.2 -- commit \texttt{928a66966c2}
``device: allocate/free Red Hat only struct device\_rh'' took the
approach of not collocating RHEL's extension structure allocations
with their respective '\texttt{device}' structure allocations, but
instead, to augment the '\texttt{device\_add()}' function to perform
the extension's allocation.

\shadowbox{\begin{minipage}[t]{\mpwidth}
\begin{lstlisting}
diff -git a/drivers/base/core.c b/drivers/base/core.c
index aa6cc6f7f2d..75ed70b3a78 100644
--- a/drivers/base/core.c
+++ b/drivers/base/core.c
@@ -1060,6 +1060,10 @@ int device_add(struct device *dev)
         if (!dev)
                 goto done;

+        /* allocate the Red Hat only struct */
+        dev->device_rh = kzalloc(sizeof(struct device_rh),
+                                 GFP_KERNEL | __GFP_NOFAIL);
+
         if (!dev->p) {
                 error = device_private_init(dev);
                 if (error)
@@ -1304,6 +1308,8 @@ void device_del(struct device *dev)
         kobject_uevent(&dev->kobj, KOBJ_REMOVE);
         cleanup_device_parent(dev);
         kobject_del(&dev->kobj);
+        /* This free's the allocation done in device_add() */
+        kfree(dev->device_rh);
         put_device(parent);
 }
\end{lstlisting}
\end{minipage}}

As device registration first obtains an allocated '\texttt{device}'
structure and then, some separate time later, calls '\texttt{device\_add()}',
there is a window of time in which 'struct device' exists, and can
be accessed, prior to device registration's completion via '\texttt{device\_add()}',
and there are driver, and kernel, paths that access members of '\texttt{struct
device}' within this window. This is not an issue for upstream, as
all accesses obviously occur after the allocation. A consequence of
the approach taken in this case for RHEL, however, is accesses within
this window that, now by extension, reference '\texttt{struct device\_rh}'.
Such accesses cause the kernel to panic de-referencing a \texttt{NULL}
pointer with RHEL as a consequence of '\texttt{struct device\_rh}'
not yet having been allocated. As such, some code paths need the extension
structure allocated prior to '\texttt{device\_add()}' being called,
which subsequently requires extra checking within RHEL's customized
'\texttt{device\_add()}' routine to see if '\texttt{struct device\_rh}'
has already been allocated or not. This was covered with commit \texttt{ef52e4a544c}
``device: add device\_rh\_alloc()''.

\shadowbox{\begin{minipage}[t]{\mpwidth}
\begin{lstlisting}
diff --git a/drivers/base/core.c b/drivers/base/core.c
index d142ffcefc2..2b210cf4b0f 100644
--- a/drivers/base/core.c
+++ b/drivers/base/core.c
@@ -1039,6 +1039,16 @@ int device_private_init(struct device *dev)
         return 0;
 }
 
+/** device_rh_alloc - allocate a device_rh
+ * @dev: device to attach device_rh to.
+ */
+void device_rh_alloc(struct device *dev)
+{
+        dev->device_rh = kzalloc(sizeof(struct device_rh),
+                                 GFP_KERNEL | __GFP_NOFAIL);
+}
+EXPORT_SYMBOL_GPL(device_rh_alloc);
+
 /**
  * device_add - add device to device hierarchy.
  * @dev: device.
@@ -1072,9 +1082,17 @@ int device_add(struct device *dev)
        if (!dev)
                goto done;

-        /* allocate the Red Hat only struct */
-           dev->device_rh = kzalloc(sizeof(struct device_rh),
-                                    GFP_KERNEL | __GFP_NOFAIL);
+        /* In general, this is where we allocate the Red Hat only struct.
+         * A known exception is the acpi_bus_scan() which requires elements
+         * from the device_rh struct during initialization.  This test will
+         * allow code to allocate & use the device_rh struct prior to calling
+         * device_add().
+         *
+         * Regardless of when allocated, the device_rh will be free'd at the
+         * end of device_del().
+         */
+        if (!dev->device_rh)
+                device_rh_alloc(dev);

         if (!dev->p) {
                 error = device_private_init(dev);
\end{lstlisting}
\end{minipage}}

The default extension related allocation remains within '\texttt{device\_add()}';
however, extension allocations necessary prior to such now have to
be enacted throughout the kernel in a somewhat ad hoc, case-by-case,
manner; usually upon each occurrence of a de-referencing panic (along
with the additional work of root cause analysis and re-working of
the path's specific, allocation related, need). \footnote{RHEL7: Since commit \texttt{ef52e4a544c} there have been similar occurrences
encountered: see commits \texttt{320d729cf76}, \texttt{3f726c5d045},
\texttt{e03b2d39fce}, \texttt{e25bbd31c16}, \texttt{a3929deae7e},
\texttt{731dc052b42}. Commit \texttt{b369cf3d0d5} is also related.} \footnote{RHEL7: With hindsight, while collocating '\texttt{struct device\_rh}'
allocations with '\texttt{struct device}' allocations would have been
much more invasive, it would have prevented the introduction of the
RHEL specific window in which references to '\texttt{struct device\_rh}'
can occur. Perhaps the invasiveness could be handled in the future
by leaving the allocations to the device drivers themselves: when
a driver recognizes the need for a new member residing within '\texttt{struct
device\_rh}', it then performs the allocation of '\texttt{struct device\_rh}'
where appropriate if it has not yet done so already.}

\shadowbox{\begin{minipage}[t]{\mpwidth}
\begin{lstlisting}
diff --git a/drivers/ata/libata-transport.c b/drivers/ata/libata-transport.c
index fd29b722408..6c52db6e941 100644
--- a/drivers/ata/libata-transport.c
+++ b/drivers/ata/libata-transport.c
@@ -660,6 +660,7 @@ static int ata_tdev_add(struct ata_device *ata_dev)
         int error;

         device_initialize(dev);
+        device_rh_alloc(dev);
         dev->parent = get_device(&link->tdev);
         dev->release = ata_tdev_release;
         if (ata_is_host_link(link))
diff --git a/drivers/base/platform.c b/drivers/base/platform.c
index 55684f06283..f6a50dbe3bd 100644
--- a/drivers/base/platform.c
+++ b/drivers/base/platform.c
@@ -209,6 +209,7 @@ struct platform_device *platform_device_alloc(const char *name, int id)
                device_initialize(&pa->pdev.dev);
                pa->pdev.dev.release = platform_device_release;
                arch_setup_pdev_archdata(&pa->pdev);
+               device_rh_alloc(&pa->pdev.dev);
        }
 
        return pa ? &pa->pdev : NULL;
diff --git a/drivers/pci/probe.c b/drivers/pci/probe.c
index 18c5a9c2b82..6fa2f418ea3 100644
--- a/drivers/pci/probe.c
+++ b/drivers/pci/probe.c
@@ -568,6 +568,7 @@ static struct pci_host_bridge *pci_alloc_host_bridge(struct pci_bus *b)
        bridge = kzalloc(sizeof(*bridge), GFP_KERNEL);
        if (!bridge)
                return NULL;
+       device_rh_alloc(&bridge->dev);
 
        INIT_LIST_HEAD(&bridge->windows);
        bridge->bus = b;
diff --git a/drivers/pnp/core.c b/drivers/pnp/core.c
index cb6ce42f8e7..0c8d8f8977f 100644
--- a/drivers/pnp/core.c
+++ b/drivers/pnp/core.c
@@ -133,6 +133,7 @@ struct pnp_dev *pnp_alloc_dev(struct pnp_protocol *protocol, int id,
        dev = kzalloc(sizeof(struct pnp_dev), GFP_KERNEL);
        if (!dev)
                return NULL;
+       device_rh_alloc(&dev->dev);
 
        INIT_LIST_HEAD(&dev->resources);
        INIT_LIST_HEAD(&dev->options);
\end{lstlisting}
\end{minipage}}

From here on out the case study is similar to \ref{sec:Case-Study:Adding-Members-Centrally}
\nameref{sec:Case-Study:Adding-Members-Centrally} with the additional
caveat that additional latent issues yet uncovered may yet occur and
need addressed as outlined.

\subsection{Caveats}

This method is not allowed after an encumbered data structure's symbol
has become protected. The extension member must be implemented prior
to becoming locked down, possibly pro-actively as part of the release's
origin.

One caveat to this method concerns bringing in future references to
any new members introduced in this fashion from upstream. As a direct
result of new members being added to RHEL's specific extension structure,
references to any such members will have to be adjusted to account
for the extra level of indirection incurred when back-porting (i.e.,
upstream references like '\texttt{dev-><member>}' have to be converted
to '\texttt{dev->device\_rh-><member>}').

Only the addition of new members is allowed. Member deletions, re-naming,
and even re-arrangements, are prohibited.

\section{Case Study: Exploiting Holes within Kernel Data Structures \label{sec:Case-Study:Exploting-Holes}}

Upstream reference: commit \texttt{9d26d3a8f1b}

RHEL7 back-port reference: commit \texttt{a040fb72b14} \footnote{RHEL7: commit \texttt{a040fb72b14} ``PCI: Put PCIe ports into D3
during suspend”}

This case study covers the easiest approach for adding members to
existing kernel data structures without breaking kABI. As the study's
title suggests, it involves exploiting existing holes within kernel
data structures. Kernel developers typically take data structure member
alignment and padding aspects into consideration during development,
attempting to both align, and pack, structures as efficiently, and
tightly, as possible. \footnote{Kernel code, especially development of kernel data structures, is
typically written to conform to natural alignment constraints, a scheme
that is sufficiently strict to ensure portability across multiple
architectures. Natural alignment requires that every \texttt{N} byte
access must be aligned on a memory address boundary of \texttt{N}
which can be expressed in terms of the modulus operator: \texttt{(addr
\% N)} must be zero.} Even so, holes within structures sometimes exist.

If an existing hole can be found within a protected kernel data structure
that, including all alignment concerns, is equal to or greater than
the size of a new member needing to be added, the hole can be exploited
for adding the member while still maintaining kABI.

An upstream patch, targeted for back-porting into RHEL, added a new
'\texttt{bridge\_d3}' member to '\texttt{struct pci\_dev}'.\textsuperscript{\ref{fn:c1f6}}

\shadowbox{\begin{minipage}[t]{\mpwidth}
\begin{lstlisting}
diff --git a/include/linux/pci.h b/include/linux/pci.h
index 8d748345b158..8597b423cb63 100644
--- a/include/linux/pci.h
+++ b/include/linux/pci.h
@@ -294,6 +294,7 @@ struct pci_dev {
      unsigned int    d2_support:1;   /* Low power state D2 is supported */
      unsigned int    no_d1d2:1;      /* D1 and D2 are forbidden */
      unsigned int    no_d3cold:1;    /* D3cold is forbidden */
+     unsigned int    bridge_d3:1;    /* Allow D3 for bridge */
      unsigned int    d3cold_allowed:1;       /* D3cold is allowed by user */
      unsigned int    mmio_always_on:1;       /* disallow turning off io/mem
                                                 decoding during bar sizing */
\end{lstlisting}
\end{minipage}}

The '\texttt{pci\_dev}' structure is the base data structure of the
kernel's PCI subsystem, and as such, there are numerous kABI-protected
symbols with associations to it. A straight back-port would break
kABI due to its side-effect of changing the version checksums of all
symbols '\texttt{pci\_dev}' is correlated with.

The new member is a single bit bit-field declaration, and with careful
inspection of RHEL's '\texttt{struct pci\_dev}' (RHEL 7's structure
varies from upstream's), there looked to be a ``hole'' between the
end of the previously existing bit-fields and the '\texttt{d3\_delay}'
member. Running the '\texttt{pahole(1)}' utility against '\texttt{struct
pci\_dev}' confirmed that a 14-bit hole existed:

\shadowbox{\begin{minipage}[t]{\mpwidth}
\begin{lstlisting}
  /* Bitfield combined with previous fields */
  unsigned int               pme_support:5;        /*   124:19  4 */
  ...
  unsigned int               bridge_d3:1;          /*   124: 7  4 */
  /* XXX 14 bits hole, try to pack */
\end{lstlisting}
\end{minipage}}

RHEL7 Patch 09/23 ``PCI: Put PCIe ports into D3 during suspend''
used just such an exploitation. During the back-port, the new member
was inserted within the existing hole. As the hole already existed,
both the size of the overall structure, and the offsets of all existing
members within the structure, were not altered as long as the insertion
was performed at the end of the bit fields. Incidentally, any existing
external modules would not know about the newly introduced member,
thus should never attempt to access that part of the structure's memory.

While the solution is understood not to logically effect existing
external modules, the change would still end up altering all symbols
with associations to '\texttt{struct pci\_dev}'. The final piece of
the overall solution is to use RHEL's '\texttt{RH\_KABI\_FILL\_HOLE()}'
macro to specifically circumvent \texttt{\_\_GENKSYMS\_\_} based checking
from detecting the change.\textsuperscript{\ref{fn:c3f10}}

\subsection{Caveats}

A benefit of this tactic is that all code references of the new member
do not have to be modified during the initial, and all future subsequent,
back-ports.

Holes within existing kernel data structures that can be utilized
are very rare. One may want to consider saving them for performance-sensitive
code additions, if another workaround is possible. If there are no
other methods available and the hole is the last one available, one
may want to consider not back-porting the code in question, instead
saving the hole for use with a more critical back-port requiring a
kABI workaround.

\section{Case Study: Kernel Header Files \label{sec:Case-Study:Kernel-Header-Files}}

Upstream reference: commit \texttt{589fcc23074}

RHEL7 back-port reference: commit \texttt{0c9beb71177} \footnote{\label{fn:b5f13}rhkernel-list: {[}RHEL7.3 PATCH 00/61{]} Re-base
PCI subsystem with v3.18 content

\texttt{0c9beb71177} {[}RHEL7.3 PATCH 15/16{]} PCI: Move pci\_configure\_slot()
to drivers/pci/probe.c}

There is an interesting situation concerning forward declarations
of kernel data structures that yields ``false kABI failures'' from
the checking tools. The scenario typically involves header files in
which prior to the patch series' changes, a source file only knows
about a specific data structure via an incomplete forward declaration
(i.e., \texttt{struct pci\_dev;}). Then, as a direct result of the
patch including an additional header file, the source file now knows
the structure's full declaration (i.e., \texttt{struct pci\_dev \{
... \};}).

Basically what has occurred with respect to the source file is that
what was an opaque reference to a structure subsequently becomes non-opaque
due to the new header file inclusion that adds -- directly, or indirectly
-- the structure's full declaration.

Including header files may change the parsing of structures, which
can result in a false-positive kABI failure.

These situations can be particularly hard to track down and isolate
as the root of the kABI failure.

During RHEL7.3's kernel development phase, commit \texttt{0c9beb71177}
`` PCI: Move pci\_configure\_slot() to drivers/pci/probe.c'' was
pro-actively back-ported, moving a function from one source file to
another.\textsuperscript{\ref{fn:b5f13}} As a necessary result of
the move, an additional header file was included within the destination's
source file.

\shadowbox{\begin{minipage}[t]{\mpwidth}
\begin{lstlisting}
diff --git a/drivers/pci/probe.c b/drivers/pci/probe.c
index 4c22517..3c01e49 100644
--- a/drivers/pci/probe.c
+++ b/drivers/pci/probe.c
@@ -6,6 +6,7 @@
 #include <linux/delay.h>
 #include <linux/init.h>
 #include <linux/pci.h>
+#include <linux/pci_hotplug.h>
 #include <linux/slab.h>
 #include <linux/module.h>
 #include <linux/cpumask.h>
\end{lstlisting}
\end{minipage}}

What was particularly interesting about this circumstance was that
RHEL's build related kABI checking did not flag a kABI failure whereas
the '\texttt{ksc(1)}' kABI checking tool, which was used by a partner,
indicated that the following symbol's kABI checksums changed and affects
(breaks) their modules, leading them to request for the symbols to
become protected:

\shadowbox{\begin{minipage}[t]{\mpwidth}
\begin{lstlisting}
pci_read_config_dword
pci_read_config_word
pci_write_config_byte
pci_read_config_byte
pci_write_config_word
pci_write_config_dword
\end{lstlisting}
\end{minipage}}

These specific symbols were not kABI white-listed (thus why this situation
was not caught during the back-port's kernel build related checking);
however, the \texttt{pci\_read/write\_config{*}} functions are really
just wrappers around \texttt{pci\_bus\_read/write\_config\_{*}} which
are kABI white-listed. \footnote{Notice the '\texttt{struct pci\_dev}' in the wrapper functions, whereas
the protected functions are using '\texttt{dev->bus}' and '\texttt{dev->devfn}'.}

This specific instance's circumstances are unique. As the symbols
were not white-listed, RHEL kernel developers were very reluctant
to add them as the partner had requested as this could be a slippery
slope that leading to RHEL being requested to preserve all the kernel's
exported symbols.

Regardless of the specific circumstances, once isolated, these types
of false-failures are generally fairly easy to rectify as they can
be worked around by the use of the \texttt{\_\_GENKSYMS\_\_} pre-processing
directive.

For this specific case, it exposed a bug in the '\texttt{ksc(1)}'
tool, so RHEL's kernel development team acquiesced, enacting the easy
fix to restore the original checksums, but in a manner explicitly
not adding the symbols to the white-list, via RHEL7 commit \texttt{0c348a6cd87}.
\footnote{RHEL7: \texttt{0c348a6cd87} ``PCI: Restore original checksums of
pci symbols''}

\shadowbox{\begin{minipage}[t]{\mpwidth}
\begin{lstlisting}
diff --git a/drivers/pci/probe.c b/drivers/pci/probe.c
index 6fa2f418ea3..cba8d30d4e6 100644
--- a/drivers/pci/probe.c
+++ b/drivers/pci/probe.c
@@ -6,7 +6,9 @@
 #include <linux/delay.h>
 #include <linux/init.h>
 #include <linux/pci.h>
+#ifndef __GENKSYMS__
 #include <linux/pci_hotplug.h>
+#endif
 #include <linux/slab.h>
 #include <linux/module.h>
 #include <linux/cpumask.h>
\end{lstlisting}
\end{minipage}}

This same technique can often be used to work around false kABI failures
with similar -- indirect forward declarations of data structures
becoming full forward declarations -- scenarios.

\subsection{Caveats}

If a header file has kABI protected structures in it, and is the only
inclusive path to an \texttt{EXPORT\_SYMBOL\{\_GPL\}()}, the \texttt{\_\_GENKSYMS\_\_}
masking scenario as outlined will now hide all future kABI check failures
that should have caught kABI protected structure changes within the
header. Masking header files in this manner should be carefully done
and may require splitting the header, or specific content therein,
in a RHEL-only manner to limit the scope of the \texttt{\_\_GENKSYMS\_\_}
wrapping.

\section{Case Study: Enumerations \label{sec:Case-Study:Enumerations}}

Upstream reference: commit \texttt{95e300c052f}

RHEL7 back-port reference: commit \texttt{a57906f6a8f} \footnote{rhkernel-list: {[}RHEL7.6 BZ 1541533 45/54 V2{]} ipmi: Make the DMI
probe into a generic platform probe}

Changes to enumeration ordering can affect external modules. RHEL's
current kABI checking capabilities may not catch simple enumeration
related changes that indeed affect modules, leading to silent kABI
failures.

Note how RHEL7.6 commit \texttt{a57906f6a8f} ``IPMI: Make the DMI
probe into a generic platform probe'' was able to maintain the values
of previously existing enumerations to preserve kABI.

\shadowbox{\begin{minipage}[t]{\mpwidth}
\begin{lstlisting}
diff --git a/include/linux/ipmi.h b/include/linux/ipmi.h
index c0c3aed..eac5879 100644
--- a/include/linux/ipmi.h
+++ b/include/linux/ipmi.h
@@ -281,7 +281,10 @@ int ipmi_validate_addr(struct ipmi_addr *addr, int len);
  */
 enum ipmi_addr_src {
        SI_INVALID = 0, SI_HOTMOD, SI_HARDCODED, SI_SPMI, SI_ACPI, SI_SMBIOS,
-       SI_PCI, SI_DEVICETREE, RH_KABI_RENAME(SI_DEFAULT, SI_LAST)
+       SI_PCI, SI_DEVICETREE, SI_DEFAULT
+#ifndef __GENKSYMS__
+       , SI_PLATFORM = SI_DEFAULT, SI_LAST
+#endif
 };
 const char *ipmi_addr_src_to_str(enum ipmi_addr_src src);
\end{lstlisting}
\end{minipage}}

Fortunately, updates of enumerations are fairly easy to adhere to
kABI's requirements. The most common tactic is to convert un-assigned
enumeration lists into explicitly assigned lists, and then if adding
new entries is needed, do so in a manner that does not affect any
previous entries (i.e., if a new, upstream, enumeration was added
within an existing list, keep the list ordered as it was and add the
new entry to the end of the list instead).

\chapter{kABI: Canonical Method for Extending Kernel Data Structures }

kABI's impact on kernel data structures is its most subtle, yet arguably
also its most severe, aspect that kernel developers must be aware
of. To circumvent this particularly onerous problem, a number of innovative
implementations have evolved over the years -- hash based associations
to shadow structures, directly associated opaque extension structures,
and pre-reserved additions with directly associated opaque extension
structures -- to name a few.

So, while there have been many iterative solutions developed over
the years, the consensus is that the method developed and used within
the kernel's networking subsystem -- a combination of pre-reserving
entries with an additional ``extension'' structure -- is considered
to be the canonical model to follow currently.

One of the key data structures within the kernel's networking subsystem
is '\texttt{net\_device\_ops}'. \footnote{include/linux/netdevice.h::\texttt{struct net\_device\_ops \{ ... \};}}
This structure is altered upstream on a consistent basis, and there
are numerous kABI-protected symbols within networking's subsystem
that reference it. Normally, any changes to '\texttt{struct net\_device\_ops}'
would break kABI (and KMI), for all symbols it is correlated with.
As such, let's analyze, via software archeology, the evolution of
enabling extensions to it over the lifecycle of RHEL 7's release,
specifically focusing on the tactics it implemented, including explanations
as to ``why'' the tactics were necessary. While there is some overlap,
it would be best to have the background knowledge, and context, from
the kernel data structure related case studies in Appendix B (\ref{sec:Case-Study:Pro-actively-Extending},
\ref{sec:Case-Study:Adding-Members-Centrally}, \ref{sec:Case-Study:Adding-Members-Non-Centrally}
\& \ref{sec:Case-Study:Exploting-Holes}) before continuing on.

As should be evident, kABI does not come into play until the originating
major release's General Availability (GA) (e.g., 7.0).\textsuperscript{\ref{fn:c3f5}}
Prior to such, external modules do not yet have a kernel basis against
which to build against, so all kernel development activities are unencumbered.
RHEL developers, especially maintainers of core kernel subsystems,
should take advantage of this possibly once-in-a-release opportunity
to get their subsystem's header files updated to upstream's 'tip'.
RHEL 7 commit \texttt{4a59d5e8211} ``{[}net{]} core: Add VF link
state control'' did just that, pulling in recent updates to '\texttt{struct
net\_device\_ops}' from upstream (commit \texttt{1d8faf48c74}) prior
to the release's launch. \footnote{RHEL7 commit \texttt{4a59d5e8211} came in with 'kernel-3.10.0-35.el7'
whereas RHEL 7.0's release was 'kernel-3.10.0-123.el7'.}

\shadowbox{\begin{minipage}[t]{\mpwidth}
\begin{lstlisting}
diff --git a/include/linux/netdevice.h b/include/linux/netdevice.h
index f7ec329dd33..553d6be0159 100644
--- a/include/linux/netdevice.h
+++ b/include/linux/netdevice.h
@@ -839,6 +839,7 @@ struct netdev_phys_port_id {
  * int (*ndo_set_vf_spoofchk)(struct net_device *dev, int vf, bool setting);
  * int (*ndo_get_vf_config)(struct net_device *dev,
  *                         int vf, struct ifla_vf_info *ivf);
+ * int (*ndo_set_vf_link_state)(struct net_device *dev, int vf, int link_state);
  * int (*ndo_set_vf_port)(struct net_device *dev, int vf,
  *                       struct nlattr *port[]);
  * int (*ndo_get_vf_port)(struct net_device *dev, int vf, struct sk_buff *skb);
@@ -1002,6 +1003,8 @@ struct net_device_ops {
        int                     (*ndo_get_vf_config)(struct net_device *dev,
                                                     int vf,
                                                     struct ifla_vf_info *ivf);
+       int                     (*ndo_set_vf_link_state)(struct net_device *dev,
+                                                        int vf, int link_state);
        int                     (*ndo_set_vf_port)(struct net_device *dev,
                                                   int vf,
                                                   struct nlattr *port[]);
\end{lstlisting}
\end{minipage}}

Subsequently, as a necessary action prior to the release's launch,
RHEL7 specific commit \texttt{259662d355d} ``net: Reserve extension
space for 'struct net\_device\_ops'`` was incorporated. This pro-active
step was enacted specifically to obviate concerns surrounding kABI
- in order that additions to '\texttt{struct net\_device\_ops}' could
occur going forward without affecting any future protected symbols
with which the structure is correlated.

Upon the occurrence of a symbol with which a structure is correlated
with becoming protected, which for '\texttt{net\_device\_ops}' is
an all but forgone conclusion, the structure becomes locked down from
any future changes if not for actions such as this. As such, it is
especially important for maintainers of externally facing core kernel
subsystems such as networking, and PCI, (to name a couple) to take
similar actions as their foundational data structures are the most
likely to become encumbered with symbols becoming protected.

\shadowbox{\begin{minipage}[t]{\mpwidth}
\begin{lstlisting}
diff --git a/include/linux/netdevice.h b/include/linux/netdevice.h
index 553d6be0159..6d6007e2df9 100644
--- a/include/linux/netdevice.h
+++ b/include/linux/netdevice.h
@@ -1078,6 +1078,32 @@ struct net_device_ops {
                                                      bool new_carrier);
        int                     (*ndo_get_phys_port_id)(struct net_device *dev,
                                                        struct netdev_phys_port_id *ppid);
+
+       /* RHEL SPECIFIC
+        *
+        * The following padding has been inserted before ABI freeze to
+        * allow extending the structure while preserve ABI. Feel free
+        * to replace reserved slots with required structure field
+        * additions of your backport.
+        */
+#ifndef __GENKSYMS__
+       void                    (*rh_reserved1)(void);
+       void                    (*rh_reserved2)(void);
+       void                    (*rh_reserved3)(void);
...
+       void                    (*rh_reserved14)(void);
+       void                    (*rh_reserved15)(void);
+       void                    (*rh_reserved16)(void);
+#endif
 };
 
 /*
\end{lstlisting}
\end{minipage}}

At the last minute, just prior to the release's GA, RHEL 7 specific
commit \texttt{023ac646f29} ``kabi: Remove GENKSYMS usage through
the code'' was enacted. \footnote{RHEL7 commit \texttt{023ac646f29} came in with 'kernel-3.10.0-120.el7'
whereas RHEL 7.0's release was 'kernel-3.10.0-123.el7'.}

\emph{...}

\emph{However folks were using this wrapper to clearly indicate which
chunks of code were being reserved for Red Hat purposes.}

\emph{After thinking about this and discussing it with folks, we realized
this is not a good idea.  Two reasons have stuck out as to why.}

\emph{First, the idea was that hiding the symbols in the GENKSYSMS
wrapper, we hide the true checksum of the symbols.  We should try
to encompass the true checksum for GA.  On the flip side, removing
the macro and doing things properly for 7.1 would entail:}

\begin{lstlisting}
struct foo {
        int bar;
#ifndef __GENKSYMS__
        int original;
#else
        int new;
#endif
}
\end{lstlisting}

\emph{which leads to ugly and hard to read code.  I agreed to create
some macros in 7.1 that would hide the ugliness, yet make it obvious
what is going on.}

\emph{Second, and this is more a social problem, most folks have come
to assume that anything inside \_\_GENKSYMS\_\_ is free to be modiifed/added.
 But in this particular example, this is \_not\_ the case.  So to
prevent the confusion, we should avoid this usage for reserving memory.}

\emph{As a result, I have removed all uses I believe were not correct.
 There should be no downside as this only affects how genksyms pre-processes
the symbols and generates checksums.  The only noticable change will
be new kabi checksums for a bunch of functions.}

\emph{...}

\begin{lstlisting}
diff --git a/include/linux/netdevice.h b/include/linux/netdevice.h
index c23d772f8b3..42e58d229b1 100644
--- a/include/linux/netdevice.h
+++ b/include/linux/netdevice.h
@@ -1105,7 +1105,6 @@ struct net_device_ops {
         * to replace reserved slots with required structure field
         * additions of your backport.
         */
-#ifndef __GENKSYMS__
        void                    (*rh_reserved1)(void);
        void                    (*rh_reserved2)(void);
        void                    (*rh_reserved3)(void);
@@ -1122,7 +1121,6 @@ struct net_device_ops {
        void                    (*rh_reserved14)(void);
        void                    (*rh_reserved15)(void);
        void                    (*rh_reserved16)(void);
-#endif
 };
 
 /*
\end{lstlisting}

With these actions (specifically commit \texttt{259662d355d}), RHEL's
networking subsystem's '\texttt{struct net\_device\_ops}' was equipped
for taking in future extensions (new members) while not affecting
(breaking) kABI. These actions are similar in effect to what case
study \fullref{sec:Case-Study:Pro-actively-Extending}
-- enacted. Such actions are predictive, based on experience, and
must be pro-active due to symbols starting to become protected and
external modules beginning to build against a just released major
version kernel.

As a result of the actions behind the context of RHEL7 commit \texttt{023ac646f29}
(too many kABI-naive engineers overusing, and abusing, \texttt{\_\_GENKSYMS\_\_};
anything inside \texttt{\_\_GENKSYMS\_\_} is not free to be modified/added),
RHEL specific kABI macros were developed, and incorporated, during
7.1's minor update development cycle {[}see: Appendix C \ref{sec:include/linux/rh_kabi.h}
\nameref{sec:include/linux/rh_kabi.h}{]}.\textsuperscript{\ref{fn:c3f5}}
Subsequently, commit \texttt{2d1a734afb0} ``kernel: Use new KABI
macros'' converted '\texttt{struct net\_device\_ops}' reserved entries
over to the new macros, specifically '\texttt{RH\_KABI\_RESERVE\_P()}'.

\shadowbox{\begin{minipage}[t]{\mpwidth}
\begin{lstlisting}
@@ -1105,22 +1107,22 @@ struct net_device_ops {
 * to replace reserved slots with required structure field
 * additions of your backport.
 */
-        void (*rh_reserved1)(void);
-        void (*rh_reserved2)(void);
-        void (*rh_reserved3)(void);
...
-        void (*rh_reserved14)(void);
-        void (*rh_reserved15)(void);
-        void (*rh_reserved16)(void);
+        RH_KABI_RESERVE_P(1)
+        RH_KABI_RESERVE_P(2)
+        RH_KABI_RESERVE_P(3)
...
+        RH_KABI_RESERVE_P(14)
+        RH_KABI_RESERVE_P(15)
+        RH_KABI_RESERVE_P(16)
 };

 /*
\end{lstlisting}
\end{minipage}}

With the kABI related reserved extension placeholders in place, along
with the inevitable upstream changes to '\texttt{struct net\_device\_ops}'
beginning to take place, exchanges such as commit \texttt{332b89c4cae}
``{[}net{]} introduce dev\_get\_iflink()'' started occurring, adding
new functionality via the new members introduced by such exchanges.

\shadowbox{\begin{minipage}[t]{\mpwidth}
\begin{lstlisting}
diff --git a/include/linux/netdevice.h b/include/linux/netdevice.h
index 239157b4ebb..02b67f7e3c4 100644
--- a/include/linux/netdevice.h
+++ b/include/linux/netdevice.h
@@ -970,6 +970,9 @@ struct netdev_phys_port_id {
  * Called by vxlan to notify the driver about a UDP port and socket
  * address family that vxlan is not listening to anymore. The operation
  * is protected by the vxlan_net->sock_lock.
+ *
+ * int (*ndo_get_iflink)(const struct net_device *dev);
+ * Called to get the iflink value of this device.
 */
 struct net_device_ops {
         int (*ndo_init)(struct net_device *dev);
@@ -1114,7 +1117,7 @@ struct net_device_ops {
 * to replace reserved slots with required structure field
 * additions of your backport.
 */
-        RH_KABI_RESERVE_P(1)
+        RH_KABI_USE_P(1, int (*ndo_get_iflink)(const struct net_device *dev))
         RH_KABI_RESERVE_P(2)
         RH_KABI_RESERVE_P(3)
         RH_KABI_RESERVE_P(4)
\end{lstlisting}
\end{minipage}}

Normally adding new functionality (members) to the structure would
affect kABI but the placeholders circumvented such unintended consequences.

The '\texttt{net\_device\_ops}' data structure exhibits a unique aspect
in that it consists exclusively of function pointers to network device
operations (NDO's): each, and every one, of its members is going to
be the same size - 8 Bytes. So later, with a back-port like \texttt{332b89c4cae}
where '\texttt{RH\_KABI\_RESERVE\_P(1)}' is exchanged for ``\texttt{int
({*}ndo\_get\_iflink)(...)}'', the conversion is exchanging 8 Bytes
(i.e., ``\texttt{unsigned long rh\_reserved\#\#n}'') with 8 Bytes,
and due to the structure's nature, exchanges are always going to be
similarly sized, thus the structure's overall size remains constant.
As such it's too easy to miss the subtle, yet important, aspects that
'\texttt{RH\_KABI\_USE\_P()}' plays in such exchanges.

In the back-port of upstream commit \texttt{125c99bc8b6} ``scsi:
add new scsi-command flag for tagged commands'' (RHEL7 commit \texttt{1a89161c09a}),
a similar kABI related exchange scenario was enacted. In this specific
case however, the exchange's resultant type was an 'int' which is
4 Bytes. \footnote{The C standard leaves it completely up to the compiler, which also
means the same compiler can make it depend on options and the target
architecture. '\texttt{gcc}', with respect to 64-bit Linux on Intel
architectures, produces 32-bit (4 Bytes) '\texttt{int}' types.}

\shadowbox{\begin{minipage}[t]{\mpwidth}
\begin{lstlisting}
diff --git a/include/scsi/scsi_cmnd.h b/include/scsi/scsi_cmnd.h
index 9b1ebe610b5..bc2c64ce04f 100644
--- a/include/scsi/scsi_cmnd.h
+++ b/include/scsi/scsi_cmnd.h
@@ -54,6 +54,9 @@ struct scsi_pointer {
         volatile int phase;
 };

+/* for scmd-flags */
+#define SCMD_TAGGED (1  0)
+
 struct scsi_cmnd {
         struct scsi_device *device;
         struct list_head list; /* scsi_cmnd participates in queue lists */
@@ -140,7 +143,7 @@ struct scsi_cmnd {
 * The following padding has been inserted before ABI freeze to
 * allow extending the structure while preserve ABI.
 */
-        RH_KABI_RESERVE_P(1)
+        RH_KABI_USE_P(1, int flags) /* Command flags */
         RH_KABI_RESERVE_P(2)
         RH_KABI_RESERVE_P(3)
         RH_KABI_RESERVE_P(4)
\end{lstlisting}
\end{minipage}}

An exchange of 8 Bytes with 4 Bytes would change the overall size
of '\texttt{struct scsi\_pointer}', resulting in a kABI failure. Perhaps
natural alignment aspects would make this work (i.e., there will be
a 4 Byte hole included after '\texttt{flags}' and before '\texttt{RH\_KABI\_RESERVE\_P(2)}')
but that is likely architecture dependent and should not be relied
upon. Careful inspection of '\texttt{RH\_KABI\_USE\_P()}' reveals
that its replacement mechanism includes putting the new member into
a union with an 'unsigned long' member which takes care of two things.
It ensures there is proper padding so the resulting size doesn't ``shrink''
and remains 8 Bytes. It also checks that the result didn't ``grow''
over 8 Bytes ('\texttt{sizeof(unsigned long)}').

So \texttt{RH\_KABI\_USE\_P()} (and \texttt{RH\_KABI\_USE()}) play
an important role that developers could easily overlook in such kABI
related exchange scenarios: they help maintain the structure's overall
size for all various exchange types so as not to inadvertently break
kABI. More specifically: From the '\texttt{check-kabi}' standpoint,
things might be fine, due to \texttt{\_\_GENKSYMS\_\_} wrappers, but
we'd actually end up with silent kABI breakage -- we'd just be lying
that kABI wasn't broken, effectively.

During RHEL 7.3's kernel development phase it was recognized that
'\texttt{struct net\_device\_ops}' was beginning to run low on kABI's
pre-reserved entries ('\texttt{RH\_KABI\_RESERVE\_P(n)}') put in place
earlier via commit \texttt{259662d355d}. As a result, commit \texttt{e33a3c136b5}
``net: introduce net\_device\_ops\_extended'' was developed.

\shadowbox{\begin{minipage}[t]{\mpwidth}
\begin{lstlisting}
diff --git a/include/linux/netdevice.h b/include/linux/netdevice.h
index e599072f4c0..750dc19d214 100644
--- a/include/linux/netdevice.h
+++ b/include/linux/netdevice.h
@@ -805,6 +805,9 @@ struct tc_to_netdev {
 };
+struct net_device_ops_extended {
+};
+
 /*
  * This structure defines the management hooks for network devices.
  * The following hooks can be defined; unless noted otherwise, they are
@@ -1286,9 +1289,25 @@ struct net_device_ops {
                                                sa_family_t sa_family,
                                                __be16 port))
         RH_KABI_RESERVE_P(15)
-        RH_KABI_RESERVE_P(16)
+        RH_KABI_USE_P(16, size_t ndo_size)
+        /* RHEL: put all new non-performance critical ndo's into
+         * net_device_ops_extended. The reserved slots above can be used
+         * only for performance critical operations.
+         * Drivers may access the extended fields directly from
+         * net_device_ops, if they allocated the net_device_ops structure
+         * themselves (usually statically). The kernel core and drivers
+         * using others' net_device_ops must access the extended fields
+         * using the get_ndo_ext macro.
+         */
+        RH_KABI_EXTEND(struct net_device_ops_extended extended)
 };

+#define get_ndo_ext(ops, field) ({                                        \
+        const struct net_device_ops *__ops = (ops);                        \
+        size_t __off = offsetof(struct net_device_ops, extended.field);        \
+        __ops->ndo_size > __off ? __ops->extended.field : NULL;                \
+        })
+
 /**
  * enum net_device_priv_flags - &struct net_device priv_flags
  *
\end{lstlisting}
\end{minipage}}

Commit \texttt{259662d355d} accomplishes a number of inter-related,
yet very subtle, aspects. Similar to what was accomplished in \ref{sec:Case-Study:Adding-Members-Non-Centrally}'s
kABI case study, a '\texttt{struct net\_device\_ops\_extended}' extension
data structure was added onto the end of '\texttt{struct net\_device\_ops()}'
via '\texttt{RH\_KABI\_EXTEND()}' which effectively masks the extension
from '\texttt{genksyms}' symbol versioning so as to not break kABI
as a unintended consequence of changing the structure's overall size.
What is different in this instance, however, is that the extension
structure is put into place after the fact (i.e., after the structure
had been locked down with respect to kABI), and is appended onto the
end of the originating structure ('\texttt{net\_device\_ops}') as
opposed to adding a pointer to a kABI specific opaque extension structure.
In this particular application, that's an important tactic to the
overall approaches' strategy as it obviates all the allocation considerations,
and subsequent resultant issues, that \ref{sec:Case-Study:Adding-Members-Non-Centrally}
suffered.

Prior to this commit, new functionality (NDOs) were enacted via substitutions
- exchanging a pre-reserved slot with a new NDO function pointer.
After this commit, similar substitutions introducing new functionality
could occur as before, or via the '\texttt{struct net\_device\_ops\_extended}'
extension structure introduced. As noted in the patch's comment, the
idea was add ``performance critical operations'' via pre-reserved
slot exchanges, and to add ``non-performance critical ndo's'' via
the newly inserted '\texttt{net\_device\_ops\_extended}' extension
structure.

A critical component of this implementation relates to its multi-faceted
guard mechanism. Extending kernel data structures correlated with
protected symbols is all about enabling new feature content developed
upstream -- via back-ports into RHEL -- without breaking RHEL's
kABI commitment. The kABI related extension mechanisms enable this,
specifically allowing new members to be added to a structure encumbered
with one or more protected symbols which can take place within any
minor update's development cycle. As a result, from the kernel's perspective:
protection encumbered data structures with extension mechanisms enabled
pro-actively may change size (growing -- from additions -- only)
during any minor update cycle.

\begin{table}[h]
\begin{centering}
\begin{tabular}{@{}l|lll@{}}
\textbf{Kernel vn} & \textbf{Num members} & \textbf{Total size} & \textbf{Member offsets}\tabularnewline
\hline 
\textbf{7.0 } & 72  & 0x240  & 0x000, 0x008, ..., 0x238\tabularnewline
\textbf{7.3 } & 79  & 0x278  & 0x000, 0x008, ..., 0x270\tabularnewline
\textbf{7.5 } & 94  & 0x2F0  & 0x000, 0x008, ..., 0x2E8\tabularnewline
\end{tabular}
\par\end{centering}
\caption{'\texttt{struct net\_device\_ops}' (kernel's perspective)}
\end{table}

However, external modules that were compiled against earlier kernel
versions, and thus would not know about such structure growth (``kernel/module
variance''), must continue to load (attach) and run upon newer kernels
- RHEL's kABI commitment.\textsuperscript{\ref{fn:c3f1}} This implies
that the kernel's data structure growth has to be enacted in a way
that accounts for this type of ``kernel/module variance''. This
is the concern that the guard mechanism involves.

As outlined in the commit's message: going forward, all NDO additions
using '\texttt{net\_device\_ops\_extended}' must have their references
``wrapped in the provided '\texttt{get\_ndo\_ext()}' macro''. Otherwise,
back-ported core kernel references to such NDO based extensions will
not even compile successfully. Additionally, again going forward,
all back-ported driver code needing to reference such NDO based extensions
must also be modified -- which includes external modules -- otherwise
they will not compile successfully. So inherent to the implementations
design, is this ``built-in'' guard mechanism's implicit facet.

The guard mechanism's other facet is explicit and has to do with its
'ndo\_size' member addition and corresponding '\texttt{get\_ndo\_ext()}'
macro which act in concert with each other. As covered in the commit's
message: ``It's intentionally a separate struct because calling anything
in the extended area is unsafe (because 'net\_device\_ops' is allocated
by drivers).'' If driver code needs to reference any of the NDO extensions,
it has to set '\texttt{ndo\_size}' as part of its allocation phase
tasks as was done in commit \texttt{b62901b5257} ``{[}net{]} move
ndo\_dfwd\_add/del\_station to net\_device\_ops\_extended''.

\shadowbox{\begin{minipage}[t]{\mpwidth}
\begin{lstlisting}
diff --git a/drivers/net/.../ixgbe/ixgbe_main.c b/drivers/net/.../ixgbe/ixgbe_main.c
index 5168b849df4..159b5bbadcc 100644
--- a/drivers/net/ethernet/intel/ixgbe/ixgbe_main.c
+++ b/drivers/net/ethernet/intel/ixgbe/ixgbe_main.c
@@ -8655,6 +8655,7 @@ ixgbe_features_check(struct sk_buff *skb, struct net_device *dev,
 }
 static const struct net_device_ops ixgbe_netdev_ops = {
+        .ndo_size = sizeof(struct net_device_ops),
         .ndo_open = ixgbe_open,
         .ndo_stop = ixgbe_close,
         .ndo_start_xmit = ixgbe_xmit_frame,
\end{lstlisting}
\end{minipage}}

Consider the scenario where an external module that is some type of
networking related driver defines within itself a '\texttt{static
struct net\_device\_ops <instance\_name>;}' structure. \footnote{The structure could also be an external variable (e.g., defined outside
of any function); but not automatic (e.g., defined within any function).
External and static variables are guaranteed to be initialized to
zero; automatic and register variables have undefined (i.e., garbage)
initial values.} From the perspective of this driver module, when compiled against
a RHEL 7.0 kernel, its local version of 'struct net\_device\_ops'
has 72 members and consumes 0x240 Bytes. Now assume that this module
has attached to, and is running on a newer kernel and calls a routine
within the networking subsystem's core, passing in as an argument,
a pointer to its copy of '\texttt{struct net\_device\_ops}' (i.e.
\texttt{<instance\_name>}). The kernel's point of view of '\texttt{struct
net\_device\_ops}' includes all the extensions that have taken place
throughout all its updates cycles. As such, code within it may attempt
to access any one of the NDO members added after 7.0's GA.

To make the scenario concrete, assume:
\begin{itemize}
\item the driver module was compiled against a 7.0 kernel;
\item the driver module attaches to, and is running on, a 7.5 kernel;
\item a routine within the kernel's networking subsystem's core, called
by the module, accesses the last function pointer added onto the '\texttt{net\_device\_ops\_extended}'
extension structure -- member 94's NDO -- via a pointer the module
defined within itself.
\end{itemize}
Without the guard mechanism in place, the resulting dereference will
not obtain what was intended, it would be accessing some completely
unrelated memory that is somewhere past the module's allocated memory
backing\linebreak'\texttt{net\_device\_ops}'. At best, chaos will ensue and
the kernel will panic. At worst, data corruption will silently occur.
This is where the commit's (\texttt{259662d355d}) guard mechanism
explicitly comes into play, circumventing this type of unintended
kernel/module variance induced occurrence.

Now consider the same scenario and concrete example, but, with the
guard mechanism -- '\texttt{ndo\_size}' and\linebreak'\texttt{get\_ndo\_ext()}'
-- enacted by commit \texttt{259662d355d} in place. Even though the
driver module has no knowledge of the new '\texttt{ndo\_size}' member
(as it was compiled against a 7.0 kernel), its idea of the corresponding
memory location backing '\texttt{net\_device\_ops}' -- '\texttt{RH\_KABI\_RESERVE\_P(16)}'
-- will contain the value '\texttt{0}' as a result of the module's
'\texttt{struct net\_device\_ops}' allocation having been '\texttt{static}'.
However, from the 7.5 kernel's perspective, the memory offset corresponds
to the '\texttt{ndo\_size}' member. As a result, when the kernel attempts
to access the member (94) via '\texttt{get\_ndo\_ext()}', the macro's
computation will fail the guard's test and return a \texttt{NULL}
value:

\shadowbox{\begin{minipage}[t]{\mpwidth}
\begin{lstlisting}
get_ndo_ext(ops, field) {
        const struct net_device_ops *__ops = (ops);        // Module's pointer
        size_t __off = offsetof(struct net_device_ops, extended.field);        // 0x2E8
        __ops->ndo_size > __off ? __ops->extended.field : NULL;        // 0x0 > 0x2E8 ?:=> NULL
}
\end{lstlisting}
\end{minipage}}

As \texttt{259662d355d}'s commit message states -- ``This is of
somehow lesser concern, as forgetting it won't cause crashes, the
pointer would just be ignored.'' -- the dereference (pointer) obtains
a \texttt{NULL} value and can be ignored.

Finally, consider the scenario, but, with respect to a driver module
that is cognizant of NDO extensions. Assume, as a concrete example:
\begin{itemize}
\item the driver module was compiled against a 7.3 kernel;
\item the driver module attaches to, and is running on, a 7.5 kernel;
\item a routine within the kernel's networking subsystem's core, called
by the module, accesses the last function pointer added onto the '\texttt{net\_device\_ops\_extended}'
extension structure that the module is aware of - member 79's NDO
-- via a pointer the module defined within itself.
\end{itemize}
When the kernel attempts to obtain member 79's NDO via '\texttt{get\_ndo\_ext()}',
the access passes the guard's test and successfully obtains the '\texttt{\_\_ops->extended.field}'
pointer:

\shadowbox{\begin{minipage}[t]{\mpwidth}
\begin{lstlisting}
get_ndo_ext(ops, field) {
        const struct net_device_ops *__ops = (ops);        // Kernel's pointer
        size_t __off = offsetof(struct net_device_ops, extended.field);        // 0x270
        __ops->ndo_size > __off ? __ops->extended.field : NULL;        // 0x278 > 0x270 ?:=> field
}
\end{lstlisting}
\end{minipage}}

As the analysis shows, there are a myriad of subtle, inter-related,
aspects involved in successfully extending kernel data structures
encumbered with protected symbols in a manner that accounts for kernel/module
variance, and that doesn't break kABI (and KMI). These actions are
similar, but expand upon such considerably, to the actions taken in
case study \ref{sec:Case-Study:Adding-Members-Non-Centrally}, and
represent the best currently known approach to date.

One size does not necessarily fit all however, so actions similar
to what were taken in case study \ref{sec:Case-Study:Adding-Members-Centrally}
may be completely adequate {[}note the networking subsystem's similar
approach that was taken in commit \texttt{112e7b85d12} ``Generalize
ndo\_gso\_check to ndo\_features\_check''{]}.

\subsection{Caveats}

With respect to substitutions (exchanges) using the pre-reserved entries:
\begin{itemize}
\item In applications similar to commit \texttt{259662d355d}, the reserved
slots above can be used only for performance critical operations.
\item All code references of the new member(s) do not have to be modified
during the initial, and all future subsequent, back-ports.
\end{itemize}
With respect to substitutions (exchanges) using the ``extension''
structure:
\begin{itemize}
\item In applications similar to commit \texttt{259662d355d}, all new non-performance
critical member(s) should be installed into the ``extension'' structure.
\item As a direct result of new members being added to RHEL's specific extension
structure, references to any such members will have to be adjusted
during the initial, and all future subsequent, back-ports.
\end{itemize}

\end{document}
